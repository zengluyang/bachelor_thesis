% !Mode:: "TeX:UTF-8"

\chapter{方案设计}
\section{B/S结构}

B/S(Browser/Server)结构\citeup{shuchun2000design}即浏览器和服务器结构。它是随着Internet技术的兴起,对C/S结构的一种变化或者改进的结构。在这种结构下,用户工作界面是通过WWW浏览器来实现,极少部分事务逻辑在前端(Browser)实现,但是主要事务逻辑在服务器端(Server)实现,形成所谓三层结构。

\subsection{传统C/S结构}
C/S(Client/Server)结构,即大家熟知的客户机和服务器结构。它是软件系统体系结构,通过它可以充分利用两端硬件环境的优势,将任务合理分配到Client端和Server端来实现,降低了系统的通讯开销。

传统的C/S体系结构虽然采用的是开放模式,但这只是系统开发一级的开放性,在特定的应用中无论是Client端还是Server端都还需要特定的软件支持。由于没能提供用户真正期望的开放环境,C/S结构的软件需要针对不同的操作系统系统开发不同版本的软件,加之产品的更新换代十分快,已经很难适应百台电脑以上局域网用户同时使用。而且代价高,效率低。

其次,采用C/S架构,网络管理工作人员既要对服务器维护管理,又要对客户端维护和管理,这需要高昂的投资和复杂的技术支持,维护成本很高,维护任务量大。

\subsection{B/S结构的优点}
与C/S结构相比较,B/S结构的优点主要有跨平台性和易维护性\citeup{jiangleihong2001mis}。

以目前的技术看,局域网建立B/S结构的网络应用,并通过Internet/Intranet模式下数据库应用,相对易于把握、成本也是较低的。它是一次性到位的开发,能实现不同的人员,从不同的地点,以不同的接入方式(比如LAN、WLAN、WAN,Internet/Intranet等)、不同的终端(PC机、智能手机以及平板电脑等)、不同的操作系统(Windows、Linux、Mac OS X、iOS、Android等)访问和操作共同的数据库;它能有效地保护数据平台和管理访问权限,服务器数据库也很安全。特别是在JAVA、PHP这样的跨平台语言出现之后,B/S架构管理软件更是方便、快捷、高效。

另外,软件系统的改进和升级越来越频繁,B/S架构的产品明显体现着更为方便的特性。对一个稍微大一点单位来说,系统管理人员如果需要在几百甚至上千部电脑之间来回奔跑,效率和工作量是可想而知的,但B/S架构的软件只需要管理服务器就行了,所有的客户端只是浏览器,根本不需要做任何的维护。无论用户的规模有多大,有多少分支机构都不会增加任何维护升级的工作量,所有的操作只需要针对服务器进行

\subsection{AJAX技术}
AJAX即“Asynchronous JavaScript and XML”(异步的JavaScript与XML技术),指的是一套综合了多项技术的浏览器端网页开发技术。Ajax的概念由Jesse James Garrett所提出\citeup{garrett2005ajax}。

传统的Web应用允许用户端填写表单(form),当提交表单时就向Web服务器发送一个请求。服务器接收并处理传来的表单,然后送回一个新的网页,但这个做法浪费了许多带宽,因为在前后两个页面中的大部分HTML码往往是相同的。由于每次应用的沟通都需要向服务器发送请求,应用的回应时间依赖于服务器的回应时间。这导致了用户界面的回应比本机应用慢得多。

与此不同,AJAX应用可以仅向服务器发送并取回必须的数据,并在客户端采用JavaScript处理来自服务器的回应。因为在服务器和浏览器之间交换的数据大量减少(大约只有原来的5\%)[来源请求],服务器回应更快了。同时,很多的处理工作可以在发出请求的客户端机器上完成,因此Web服务器的负荷也减少了。

通过Ajax技术,采用B/S架构的程序也能在客户端电脑上进行部分处理和刷新,从而大大的减轻了服务器的负担;并增加了交互性,能进行局部实时刷新,能够达到和传统C/S(客户端-服务端)架构相同的用户体验。