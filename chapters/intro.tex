% !Mode:: "TeX:UTF-8"

\chapter{引言}
\section{选题背景及意义}
随着互联网技术的兴起,越来越多的信息管理、办公自动化、内容管理系统采用B/S架构实现,即浏览器-服务器架构。是Web兴起后的一种网络结构模式,Web浏览器是客户端最主要的应用软件。这种模式统一了客户端,将系统功能实现的核心部分集中到服务器上,简化了系统的开发、维护和使用。客户机上只要安装一个浏览器,服务器安装Oracle、Sybase、Informix、MySQL或SQL Server等数据库,浏览器通过Web Server 同数据库进行数据交互。B/S最大的优点就是可以在任何地方进行操作而不用安装任何专门的软件。只要有一台能上网的电脑就能使用,客户端零维护。系统的扩展非常容易。它具有跨平台、分布性特点,业务扩展简单、维护方便。

随着传统管理信息系统的功能复杂性不断越来越高,以及用户对系统易用性、易操作性要求不团提高,传统B/S模式架构的局限性越来越明显。Ajax(Asynchronous JavaScript and XML)\citeup{garrett2005ajax}技术的发展为解决这种局限性的方法指明了一个方向。
通过Ajax技术,采用B/S架构的程序也能在客户端电脑上进行部分处理和刷新,从而大大的减轻了服务器的负担;并增加了交互性,能进行局部实时刷新,能够达到和传统C/S(客户端-服务端)架构相同的用户体验。
在本课题中拟通过设计一个团队网络信息管理系统,来契合这两大需求,提高科研团队内交流、管理以及科研的效率,并能通过对系统中的数据的搜索、筛选、统计、分析,给科研团队的决策提供数据上的支撑。

\section{论文组织结构}

本文一共分为六章。本章介绍了团队信息网络管理系统的设计与实现课题背景和意义。

第二章~理论基础:介绍了设计和实现本系统所需要的理论知识与技术基础。

第三章~需求分析:介绍了本系统需要实现的功能。

第四章~方案设计:介绍本系统设计过程中的技术选型以及如何使用这些技术来完成本系统的设计

第五章~系统实现:介绍了本系统的实现过程,包括了:数据库实现、模型层实现、视图层实现以及控制层实现。

第六章~系统测试:对本系统进行了功能测试和性能测试,针对测试结果进行分析。

第七章~结束语:总结与展望这次的课题。