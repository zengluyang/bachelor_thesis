% !Mode:: "TeX:UTF-8"

\chapter{结束语}
在课题的方案设计和详细实现中,给出了系统软件的结构和详细设计方法。
同时也用在文中提到的开发环境实现了绝大多数的需求,并且经过多次测试,各
个功能均能正常运行且没有漏洞,整个系统具有完善的功能、良好的用户体验以及可维护的代码,都达到了预期的期望和课题要求。总的来说本系统具有以下优点:

\begin{itemize}
\item 易用性:本系统在各个模块中支持~Excel~数据的批量导入;在下拉菜单中只是拼音首字母的快速筛选;能够各种终端屏幕大小的不同的显示不同的布局。都是考虑到了本系统的操作与使用应该人性化。
\item 易维护性:本系统的基础构件简单,易于调整,基于MVC框架进行设计,各模块独立,代码风格良好,易于维护和增加新的功能。
\item 性能较好:根据本系统性能测试的结果,本系统部署在家用PC上,每秒钟能够处理50个请求,响应时间在毫秒级,具有较好的并发性与响应速度,若是部署在专用的服务器上,远远达到了本系统日常使用过程中对并发访问以及响应速度的要求。
\end{itemize}

% 作者也通过本课题的研究、学习、设计和程序实现,了解了软件工程的主要
% 的设计方法和流程,也通过学习资料掌握了基本的Web开发知识。在这个过程中,
% 体会到了想开发出一个优秀的Web应用需要在各方面进行精心设计,才能做出一
% 个开发者和用户都满意的应用。

在本系统的设计与实现中,经历了软件开发过程中完整的瀑布模型\footnote{瀑布模型(或称瀑布式开发流程)是由W.W.Royce在1970年首次提出的软体开发模型,在瀑布模型中,软件开发被分为需求分析,设计,实现,测试 (确认), 集成,和维护这样的步骤依序进行。},学习和掌握了软件开发的基本流程和方法。同时也通过学习Web开发知识,掌握了Web开发的基本流程和最佳实践。

当然,在未来本系统的实际上线使用过程中,难免会出现一些功能上以及性能的瓶颈,本系统还需要不断地改进与完善。