% !Mode:: "TeX:UTF-8"
\chapter{系统需求分析}

本章主要完成了整个系统的需求分析,主要包括了各个模块需要实现的功能,只有明确了需求,才能正确地实现本系统,对整个系统具有决策性、方向性、策略性的作用。

\section{总体需求}
为科研团队开发、设计一个“团队信息网络管理系统”,目标是建立及时、准确、全面的科研团队信息管理平台。通过与系统使用者进行沟通与调研,完成了本系统的需求分析。

本系统的目的针对科研团队信息管理的实际情况,全面覆盖科研团队在学术论文、科研项目、专利、人员管理等多个方面,并提供对这些数据的搜索、筛选以及分析功能,提高科研团队的工作以及沟通效率,并且为科研团队的决策、考核提供有力的支撑。同时,本系统需要利用这些数据,将学术论文、科研项目、专利等作为内容对外展示,方便其他人对团队进行了解和交流。本系统的总体功能框图如图~\ref{overallarc.pdf}~所示,其中科研项目、学术论文、专利管理部分需要处理动态数据,本文将主要描述它们的设计与实现,其它部分都是静态内容,非常简单,在本文中不再赘述。

\pic[H]{系统总体功能框图}{width=1.0\textwidth}{overallarc.pdf}

\subsection{功能性需求}

本系统的基本功能是实现科研团队信息的对外展示和内部管理,围绕这个功能,可将本系统的用户分为两种角色,一种角色是游客,另一种是本系统的管理员。游客只能访问每个数据管理模块的对外展示功能;而系统管理员可以对各项数据进行管理。因此,本系统应具有以下管理功能:


% \begin{description}
%   \item[学术论文管理] 管理员可以录入、增加、删除、修改、筛选、格式化导出学术论文
%   \item[专利管理] 管理员可以录入、增加、删除、修改、筛选、格式化导出专利
%   \item[科研项目管理] 管理员可以录入、增加、删除、修改、筛选、格式化导出科研项目
%   \item[人员管理] 管理员可以录入、增加、删除、修改、筛选、格式化导出人员
% \end{description}
\begin{enumerate}
\item 管理员可以对各项数据进行录入、删除、修改、查询。
\item 管理员可以对数据进行多条件搜索、筛选。
\item 管理员可以对数据进行格式化地导入和导出。
\item 管理员可以设置
\item 游客可以查看各项数据对外展示的部分
\end{enumerate}

另外考虑到在系统使用过程中,会有多个管理员对本系统所管理的数据进行维护,还应该进行对数据的分级权限管理,将管理员划分为超级管理员和子管理员,子管理员只能对某项或某几项特定的数据进行维护。

本系统的系统总体用例图如图~\ref{system_arc2.pdf}~所示:

\pic[H]{系统用例图}{width=0.8\textwidth}{system_arc2.pdf}


\subsection{易用性需求与维护性}

本系统应该对用户来说满足易用,比如用户初次使用本系统时,应该能够批量的录入数据。

易用性的高低决定着用户对产品的第一印象是好是坏。因为本系统需要向用户展示大量的数据,于是用户需要一个简洁、易懂的界面,这样他们就能不需要任何帮助地一眼获得需要的信息。

另外,应该能够支持多终端对本系统的访问和使用,根据PC、平板电脑、智能手机等终端普通的屏幕大小和分辨率,采用不同的界面布局,提高系统的用户友好性。

考虑到科研团队的一般规模,应该能够支持200个用户的并发访问,以应对本系统多人同时使用的情形。另外,本系统在各种操作和处理中响应速度应为秒级甚至毫秒级,达到实时要求。尽量减少卡顿情况。

一个软件难免需要人来维护,或者需要对功能进行升级。很多时候,维护或升级的人员因为原先的代码可读性差以致无法顺利地进行进一步操作。因此,系统的程序源代码应该具备足够的易读性,具备标准的格式和简洁的代码风格,这样能使维护和升级顺利地进行。同时开发者还需提供详细的开发文档、部署文档以及使用说明。



% 对于权限控制模块,需要实现以下功能:

% \begin{enumerate}
% \item 对用户进行管理,实现管理员的建立和其它用户的注册、删除、修改功能
% \item 划分用户的权限级别,对某个用户是否能够操作某个数据管理模块进行认证和鉴权
% \end{enumerate}
\section{典型模块需求}

限于篇幅所限,对于四个数据管理子模块,本文仅选取数据表项最为复杂、功能最多的学术论文管理模块和科研项目管理模块进行需求分析;专利管理模块以及人员管理模块数据表项与功能相对简单,且在功能上与学术论文管理模块和科研项目管理模块相类似,在本文中,仅给出它们的数据表项。在设计与实现的过程中,完成了全部模块。
\subsection{学术论文管理需求}

通过将科研团队已发表和拟发表的学术论文录入到本系统中进行管理,可以及时地跟踪学术论文的发表状态,准确地根据作者、时间段、类别、支撑项目等不同条件搜索、筛选以及导出特定的学术论文,供科研团队里面的老师和学生进行使用,为团队的决策、考核提供有力的数据支撑。

学术论文管理模块主要要实现以下几个功能。

\subsubsection{数据表项}
需要设计一个数据库表,存储以下学术论文数据:
\begin{enumerate}
\item 学术论文信息,如:“Xiaoyan Huang, Yuming Mao, Fan Wu, ``Low Complexity Utility-based Scheduling Algorithm for Heterogeneous Services in OFDM Wireless Networks,'' In: Proc. of ICCCAS 2009, San Jose, USA, vol 1, pp.48-52.”
\item 作者:按顺序存储第一作者至第五作者
\item 状态:录用待发、已发表或已检索
\item 时间:录用时间、发表时间和检索时间
\item 检索类型:SCI、EI和ISTP的检索号
\item 学术论文级别:一级、核心、其他刊物、期刊、会议、国际、高水平
\item 学术论文文件:存储学术论文对应的PDF或者Microsoft Word文档
\item 支柱项目:数个支柱项目
\item 报账项目:数个报账项目
\end{enumerate}

\subsubsection{录入}
分别需要实现对学术论文数据的批量录入和逐个录入。

批量录入指用户通过上传指定格式的Microsoft Excel表格,批量地将学术论文数据导入到数据库中。在批量录入中,需要实现替换已有表项和对已有表项进行添加的功能。

逐个录入:通过一个表单,提示用户输入或者选择以下内容:

\begin{enumerate}
\item 学术论文维护人员:从人员模块获取人员的姓名,生成下拉菜单进行选择,也可以通过输入拼音首字母快速选择
\item 学术论文信息
\item 作者1——作者5:与选择学术论文维护人员时相同
\item 状态:在“录用待发”、“已发表”、“已检索”中三选一
\item 时间:
\begin{itemize}
	\item 若状态为录用待发,则显示输入“录用时间”的表单
	\item 若状态为已发表,则显示输入“发表时间”的表单
	\item 若状态为已检索,则显示输入“发表时间”和“检索时间”的表单
\end{itemize}
\item 检索类型:仅当状态为已检索时,需要用户输入,提示用户输入SCI、EI或ISTP的检索号
\item 发表级别:在会议、期刊、国际、一级、核心、其他期刊中勾选一项或者多项
\item 支柱项目:在科研项目的列表中选择,也可以通过拼音首字母搜索
\item 报账项目:与选择支柱项目时相同
\item 高水平:在“是”、“否”中二选一
\item 学术论文文件:提示用户选择学术论文对应的pdf文件或者Microsoft Word文件进行上传
\end{enumerate}

\subsubsection{修改与删除}

对已有表项进行修改与删除,能够先按条件搜索到需要修改或删除的学术论文,然后对相关项修改或删除。学术论文修改功能的用户界面与流程与学术论文录入的用户界面与流程一致,只是表单各条目需要显示已有数据。学术论文删除功能需要实现将选中的学术论文条目从存储学术论文的数据库表中删除。


\subsubsection{显示}

学术论文的显示功能需要分为对内显示和对外显示。

对外显示:仅选择类型为高水平的学术论文,具体功能如下:

\begin{enumerate}
\item 排序原则:最新日期置顶
\item 显示内容:数据表项中的学术论文信息
\item 显示方式:表格分页形式
\end{enumerate}

对内显示:需要显示所有的学术论文,显示格式与对外显示一致。另外,需要在页面上醒目位置设置“查询”按钮,通过点击“查询”按钮,进入查询和导出功能。

\subsubsection{查询与导出}

需要可按单条件和组合条件查询,以表格形式显示并以Excel表格形式导出(显示和导出内容相同)。单条件包括:维护者、作者、状态、时间、检索类型、发表级别、支柱项目、报账项目。组合条件即是以上2个及以上条件的组合。

由于篇幅所限,本文在这里仅给出“按维护者查询”、“按作者查询”和“按时间段查询”的需求,其它条件、以及多条件查询的需求类似在这里给出的三种需求。

按维护者查询:排序按时间,最新时间置顶(录用、发表时间统一考虑),如时间相同,则以已检索、已发表、已录用再排。显示以及导出的格式如表~\ref{paper_export_maintainer_tbl}~所示,其中:时间一项要求只录用的为录用时间,其它显示为发表时间;检索一项需要包含检索类型和检索号。导出的Excel文件名需要命名为“由XXX维护的学术论文”。

\threelinetable[htbp]{paper_export_maintainer_tbl}{1.0\textwidth}{ccccc}{学术论文按维护者查询导出格式}
{序号&学术论文信息&状态&时间&检索\\
}{
1&Xiaoyan Huang, Yuming Mao...&已检索&2011.01.01&EI:20111713930271\\
2&黄晓燕,毛玉明,吴凡,冷甦鹏.,“基于...&已发表&2009.05.01&~\\
}{}

按作者查询:排序按时间,最新时间置顶(录用、发表时间统一考虑),如时间相同,则以已检索、已发表、已录用再排。导出的Excel文件名需要命名为“XXX发表的学术论文”。

按时间段查询:如2000年之后、2012年9月至2013年6月等,排序按时间,最新时间置顶(录用、发表时间统一考虑),如时间相同,则以已检索、已发表、已录用再排。导出的Excel文件名需要命名为“2000年至2013年之间发表的学术论文”。

\subsection{科研项目管理需求}
\label{projectdemand}

与学术论文管理模块类似,通过将科研项目的信息录入到本系统,可以及时地跟踪各个项目的当前状态和信息,按照人员、年份、级别等不同条件对科研项目进行搜索、筛选和显示,供科研团队里面的老师和学生进行使用,为团队的决策、考核提供有力的数据支撑。

\subsubsection{数据表项}

需要设计一个数据库表,存储以下信息:

\begin{enumerate}
\item 维护人员
\item 项目名称
\item 项目编号
\item 经本费编号
\item 级别:国际级、国家级、省部级、市级、校级、横向、国家自然基金、973、863、科技支撑计划、教育部高校博士点基金、重大专项和XX项目中的一项或者多项
\item 开始时间
\item 截止时间
\item 结题时间
\item 申报时间
\item 立项时间
\item 申报经费
\item 立项经费
\item 实际执行人员
\item 责任书人员:数个责任书人员
\end{enumerate}

\subsubsection{录入}
\label{projectcreate}
同学术论文管理模块类似,分别需要实现对学术论文数据的批量录入和逐个录入。

批量录入指用户通过上传指定格式的Microsoft Excel表格,批量地将学术论文数据导入到数据库中。在批量录入中,需要实现替换已有表项和对已有表项进行添加的功能。

在逐个录入中,通过一个表单,提示用户输入或者选择以下内容:

\begin{enumerate}
\item 维护人员:从人员模块获取人员的数据,生成下拉菜单进行选择,也可以通过输入拼音首字母快速选择
\item 项目名称
\item 项目编号
\item 经本费编号


\item 级别:在国际级、国家级、省部级、市级、校级、横向、国家自然基金、973、863、科技支撑计划、教育部高校博士点基金、重大专项和XX项目中勾选一项或者多项
\item 时间:开始时间、截止时间、结题时间、申报时间、立项时间
\item 经费:申报经费、立项经费
\item 实际执行人员:在人员的列表中进行多项选择,也可以通过输入拼音首字母快速选择
\item 责任书人员:与实际执行人员的录入类似
\end{enumerate}

\subsubsection{显示}
\label{projectview}
对外显示:要求校级项目和XX项目不显示,申报项目不显示。排序原则:项目级别和类型按时间排序(最新截至时间置顶),级别类型顺序如下级别类型从高到低:国际合作,国家自然基金,国家973,国家863,国家科技支持计划,重大专项,省部级,市级,横向。显示格式需要以表格的形式进行显示。

对内显示:研究项目分为责任版和执行版(主要是人员可能有差异)两个菜单可选。选择版本后,其显示栏目完全相同。栏目内容都显示。级别类型中,将XX项目放在最后。默认显示所有数据表项。对于申报项目,默认显示的数据表项:项目名称、级别、类型、申报时间、立项时间、人员。查询时置顶按相关条件。如按人员查询,则按其排名先后和时间后先(排名,最新时间),等等。

\subsubsection{修改与删除}
\label{projectupdate}
对已有表项进行修改与删除,能够先按条件搜索到需要修改或删除的科研项目,然后对相关项修改或删除。科研项目修改功能的用户界面与流程与科研项目录入的用户界面与流程一致,只是表单各条目需要显示已有数据。科研项目删除功能需要实现将选中的科研项目条目从存储科研项目的数据库表中删除。

\subsubsection{查询}
\label{projectsearch}
需要可按单条件和组合条件查询,以表格形式显示并以Excel表格形式导出(显示和导出内容相同)。单条件包括:维护者、实际参与人员、责任书参与人员、级别、时间等。组合条件即是以上2个及以上条件的组合。

\section{其它数据管理需求}
\subsection{专利管理需求}

专利管理模块的数据表项:

\begin{enumerate}
\item 维护人
\item 专利名称
\item 申请时间
\item 申请号
\item 授权时间
\item 授权号
\item 级别:国际或国内二选一
\item 类型:发明专利或实用新型专利二选一
\item 发明人:数个(少于五个)发明人
\item 报账项目
\end{enumerate}

\subsection{人员管理需求}

\begin{enumerate}
\item 中文名
\item 英文名
\item 类型:学生或教师二选一
\item 介绍
\end{enumerate}

\section{权限管理需求}

在本系统中,需要区分游客和内部人员,以及将内部人员分为超级管理员和子模块管理员。

\subsection{数据表项}
需要设计一个数据库表,存储以下权限管理数据

\begin{enumerate}
\item 用户名
\item 密码散列值
\item 权限等级:系统超级管理员、学术论文模块管理员、科研项目模块管理员、专利管理员或人员管理员中的一个或者多个
\item 电子邮箱:用于忘记密码后的找回
\end{enumerate}

\subsection{权限的分级配置}

在系统第一次部署时,需要提示当前用户(系统的部署者)输入一个电子邮箱地址,生成一个用户名为“admin”密码为随机生成的用户作为系统超级管理员,并将这些信息显示和发送给输入的电子邮箱地址。

有了系统超级管理员之后,管理员可以按需的查看、增加、删除、修改用户,并且为这些用户配置各个数据管理模块的管理权限。

\subsection{用户管理}

查看用户:按表格的形式列出当前系统所有用户,在每个用户条目位置显示增加、删除、修改的链接。

增加、修改用户:通过一个表单,提示管理员输入或选择以下内容:

\begin{enumerate}
\item 用户名:要求只能是英文字母和下划线的组合
\item 密码和重复密码:若密码和重复密码相符合,将密码的散列值存储在数据库中,不能存储明文密码,以免泄漏用户隐私
\item 权限等级:在系统超级管理员、学术论文模块管理员、科研项目模块管理员、专利管理员或人员管理员中的一个或者多个
\item 电子邮箱
\end{enumerate}

