% !Mode:: "TeX:UTF-8"

\chapter{详细设计与实现}

本章给出了实现本系统详细设计与实现的过程,阐述了开发环境的搭建和配置、数据库的设计与实现、视图层的设计与实现以及模型层的设计与实现,最终完成了各个功能模块的软件开发。

\section{开发环境的搭建和配置}

\subsection{开发服务器搭建}

在实现本系统之前,需要搭建出第~\pageref{lamp}~页的第~\ref{lamp}~节提出的LAMP服务器环境,然后在此开发环境下进行开发与调试。
首先安装~Ubuntu 12.04 LTS~操作系统,Ubuntu是一个以桌面应用为主的GNU/Linux操作系统。打开终端,输入
sudo apt-get install tasksel, 安装tasksel。tasksel是一个Debian下的安装任务套件,如果你为了使你的系统完成某一种常规功能,而需要安装多个软件包时,可以使用它进行方便快捷的安装。安装成功后打开tasksel,如图~\ref{tasksel}~所示,选择LAMP Server,一个默认配置的LAMP服务器便搭建配置完毕了。
\pic[hbtp]{利用tasksel工具安装配置LAMP开发服务器}{width=1.0\textwidth}{tasksel}
此时,在浏览器中访问~http://localhost~就能够打开一个标题为“It works!”的默认网页,说明开发服务器已经正常运行了。 

\subsection{版本控制工具}
git是一个版本控制系统,用来保留工程源代码历史状态的命令行工具。可以利用它来追踪项目中的文件,并且得到某些时间点提交的项目状态。通过使用git,可以方便地在开发本系统的过程中,任意回溯到源码的不同版本上,提高开发效率,保证了源码库的安全。图~\ref{git}~展示了在开发本系统过程中的部分git日志,其中每个条目代表对代码库的一次提交,可以在不同的提交之间回溯。
\pic[hbtp]{本系统开发过程中git的部分日志}{width=0.6\textwidth}{git}
\section{数据库设计与实现}
\subsection{数据库引擎选择}

InnoDB和MyISAM是MySQL中最常用的两个数据库引擎。~MyISAM~是~MySQL~关系数据库管理系统的默认储存引擎。这种MySQL表存储结构从旧的ISAM代码扩展出许多有用的功能。InnoDB是MySQL的另一个存储引擎,正成为目前MySQL AB所发行新版的标准,被包含在所有二进制安装包里。较之于其它的存储引擎它的优点是它支持兼容ACID的事务(类似于PostgreSQL),以及参数完整性(对外键的约束)。在新版本的MySQL中,InnoDB引擎由于其对事务,参照完整性,以及更高的并发性等优点开始广泛的取代MyISAM。

考虑到在本系统中,需要在各个数据管理模块的数据库表中保存一个维护者字段,在科研项目管理模块中为每个科研项目记录保存不定数量的人员,在学术论文管理模块中为每个项目记录保存不定数量的作者,以及在专利管理模块中为每个专利记录保存不定数量的发明人,需要使用到外键和创建额外的关系数据表,详见第页~\pageref{relation}~第~\ref{relation}~节所介绍的数据表关系;考虑到InnoDB对于参数完整性的支持能够很好地保证本系统数据的一致性和完整性,本系统中的所有数据库表选用InnoDB引擎。在MySQL中指定数据库引擎非常简单,只需要在创建数据库表的SQL语句中指定“ENGINE=InnoDB”

\subsection{数据库字符集选择}
\label{utf8}
字符集是一套符号和编码的规则,不论是在Oracle数据库还是在MySQL数据库,都存在字符集的选择问题,而且如果在数据库创建阶段没有正确选择字符集,那么可能在后期需要更换字符集,而字符集的更换是代价比较高的操作,也存在一定的风险。

UTF-8(8-bit Unicode Transformation Format)是一种针对Unicode的可变长度字符编码,是用以解决国际上字符的一种多字节编码,它对英文使用8位(即一个字节),中文使用24位(三个字节)来编码。UTF-8包含全世界所有国家需要用到的字符,是国际编码,通用性强。

考虑到通用性和易用性,本系统中的所有数据库表选用UTF-8字符集。在MySQL中指定数据库字符集同样非常简单,只需要在创建数据库表的SQL语句中指定“DEFAULT CHARSET=utf8”


\subsection{数据库表项设计与实现}
根据需求分析中提出的各数据管理模块需要实现存储的表项,在MySQL中分别创建了学术论文、专利、科研项目、人员四个数据库表。

\subsubsection{学术论文}

\threelinetable[H]{paperdatabase}{0.575\textwidth}{lcr}{tbl\_paper表的结构}
{字段名称&数据类型&说明\\
}{
id&整型&主键,自动编号\\
info&文本&学术论文信息,不能为空\\
status&整形&状态\\
pass\_date&日期&录用时间\\
pub\_date&日期&发表时间\\
index\_date&日期&检索时间\\
sci\_number&变长字符串&SCI检索号\\
ei\_number&变长字符串&EI检索号\\
istp\_number&变长字符串&ISTP检索号\\
is\_first\_grade&布尔&是否一级\\
is\_core&布尔&是否核心\\
is\_journal&布尔&是否期刊\\
is\_conference&布尔&是否会议\\
is\_intl&布尔&是否国际\\
is\_domestic&布尔&是否国内\\
file\_name&变长字符串&学术论文文件名\\
file\_type&变长字符串&学术论文文件类型\\
file\_content&二进制数据&学术论文文件数据\\
is\_high\_level&布尔&是否高水平\\
maintainer\_id&整形&维护人员id\\
}{}
表~\ref{paperdatabase}~给出了学术论文数据库表的结构,其中学术论文信息比较长,可能超过255个字符,因此采用MySQL中的“mediumtext”数据类型,其他变长字符串均采用“varchar(255)”数据类型,日期采用“date”数据类型,主键id与外键维护人员id采用“int(11)”数据类型,是否一级、是否核心等布尔字段均采用“tinyint(1)”型数据类型。综合上述各字段数据类型的选择,使用下面的SQL语句,在MySQL数据库中生成学术论文管理模块的数据库表。


\noindent
\ttfamily
\hlstd{CREATE\ TABLE\ IF\ NOT\ EXISTS\ \textasciigrave tbl\textunderscore paper\textasciigrave \ (\hspace*{\fill}\\
}\hlstd{\ \ }\hlstd{\textasciigrave id\textasciigrave \ int(}\hlnum{11}\hlstd{)\ NOT\ NULL\ AUTO\textunderscore INCREMENT,\hspace*{\fill}\\
}\hlstd{\ \ }\hlstd{\textasciigrave info\textasciigrave \ mediumtext\ COLLATE\ utf8\textunderscore bin\ NOT\ NULL,\hspace*{\fill}\\
}\hlstd{\ \ }\hlstd{\textasciigrave status\textasciigrave \ tinyint(}\hlnum{4}\hlstd{)\ DEFAULT\ NULL,\hspace*{\fill}\\
}\hlstd{\ \ }\hlstd{\textasciigrave pass\textunderscore date\textasciigrave \ date\ DEFAULT\ NULL,\hspace*{\fill}\\
}\hlstd{\ \ }\hlstd{\textasciigrave pub\textunderscore date\textasciigrave \ date\ DEFAULT\ NULL,\hspace*{\fill}\\
}\hlstd{\ \ }\hlstd{\textasciigrave index\textunderscore date\textasciigrave \ date\ DEFAULT\ NULL,\hspace*{\fill}\\
}\hlstd{\ \ }\hlstd{\textasciigrave sci\textunderscore number\textasciigrave \ varchar(}\hlnum{255}\hlstd{)\ COLLATE\ utf8\textunderscore bin\ DEFAULT\ NULL,\hspace*{\fill}\\
}\hlstd{\ \ }\hlstd{\textasciigrave ei\textunderscore number\textasciigrave \ varchar(}\hlnum{255}\hlstd{)\ COLLATE\ utf8\textunderscore bin\ DEFAULT\ NULL,\hspace*{\fill}\\
}\hlstd{\ \ }\hlstd{\textasciigrave istp\textunderscore number\textasciigrave \ varchar(}\hlnum{255}\hlstd{)\ COLLATE\ utf8\textunderscore bin\ DEFAULT\ NULL,\hspace*{\fill}\\
}\hlstd{\ \ }\hlstd{\textasciigrave is\textunderscore first\textunderscore grade\textasciigrave \ tinyint(}\hlnum{1}\hlstd{)\ DEFAULT\ NULL,\hspace*{\fill}\\
}\hlstd{\ \ }\hlstd{\textasciigrave is\textunderscore core\textasciigrave \ tinyint(}\hlnum{1}\hlstd{)\ DEFAULT\ NULL,\hspace*{\fill}\\
}\hlstd{\ \ }\hlstd{\textasciigrave other\textunderscore pub\textasciigrave \ varchar(}\hlnum{255}\hlstd{)\ COLLATE\ utf8\textunderscore bin\ DEFAULT\ NULL,\hspace*{\fill}\\
}\hlstd{\ \ }\hlstd{\textasciigrave is\textunderscore journal\textasciigrave \ tinyint(}\hlnum{1}\hlstd{)\ DEFAULT\ NULL,\hspace*{\fill}\\
}\hlstd{\ \ }\hlstd{\textasciigrave is\textunderscore conference\textasciigrave \ tinyint(}\hlnum{1}\hlstd{)\ DEFAULT\ NULL,\hspace*{\fill}\\
}\hlstd{\ \ }\hlstd{\textasciigrave is\textunderscore intl\textasciigrave \ tinyint(}\hlnum{1}\hlstd{)\ DEFAULT\ NULL,\hspace*{\fill}\\
}\hlstd{\ \ }\hlstd{\textasciigrave is\textunderscore domestic\textasciigrave \ tinyint(}\hlnum{1}\hlstd{)\ DEFAULT\ NULL,\hspace*{\fill}\\
}\hlstd{\ \ }\hlstd{\textasciigrave file\textunderscore name\textasciigrave \ varchar(}\hlnum{255}\hlstd{)\ COLLATE\ utf8\textunderscore bin\ DEFAULT\ NULL,\hspace*{\fill}\\
}\hlstd{\ \ }\hlstd{\textasciigrave file\textunderscore type\textasciigrave \ varchar(}\hlnum{255}\hlstd{)\ COLLATE\ utf8\textunderscore bin\ NOT\ NULL,\hspace*{\fill}\\
}\hlstd{\ \ }\hlstd{\textasciigrave file\textunderscore size\textasciigrave \ int(}\hlnum{11}\hlstd{)\ NOT\ NULL,\hspace*{\fill}\\
}\hlstd{\ \ }\hlstd{\textasciigrave file\textunderscore content\textasciigrave \ mediumblob\ NOT\ NULL,\hspace*{\fill}\\
}\hlstd{\ \ }\hlstd{\textasciigrave is\textunderscore high\textunderscore level\textasciigrave \ tinyint(}\hlnum{1}\hlstd{)\ DEFAULT\ NULL,\hspace*{\fill}\\
}\hlstd{\ \ }\hlstd{\textasciigrave maintainer\textunderscore id\textasciigrave \ int(}\hlnum{11}\hlstd{)\ DEFAULT\ NULL,\hspace*{\fill}\\
}\hlstd{\ \ }\hlstd{PRIMARY\ KEY\ (\textasciigrave id\textasciigrave ),\hspace*{\fill}\\
}\hlstd{\ \ }\hlstd{KEY\ \textasciigrave tbl\textunderscore paper\textunderscore ibfk\textunderscore 1\textasciigrave \ (\textasciigrave maintainer\textunderscore id\textasciigrave )\hspace*{\fill}\\
)\ ENGINE=InnoDB}\hlstd{\ \ }\hlstd{DEFAULT\ CHARSET=utf8\ COLLATE=utf8\textunderscore bin;}\hspace*{\fill}\\
\mbox{}
\normalfont
\normalsize


\subsubsection{科研项目}
\label{project}
\threelinetable[H]{projectdatabase}{0.625\textwidth}{lcr}{tbl\_project表的结构}
{字段名称&数据类型&说明\\
}{
id&整型&主键,自动编号\\
name&变长字符串&项目名称\\
number&变长字符串&编号\\
fund\_number&变长字符串&经费本编号\\
is\_intl&布尔&是否国际\\
is\_national&布尔&是否国家级\\
is\_provincial&布尔&是否省部级\\
is\_city&布尔&是否市级\\
is\_school&布尔&是否校级\\
is\_enterprise&布尔&是否横向\\
is\_NSF&布尔&是否国家自然基金\\
is\_973&布尔&973\\
is\_863&布尔&863\\
is\_NKTRD&布尔&是否科技支撑计划\\
is\_DFME&布尔&是否教育部博士点专项基金\\
is\_major&布尔&是否重大专项\\
start\_date&日期&开始时间\\
deadline\_date&日期&截至时间\\
conclude\_date&日期&结题时间\\
app\_date&日期&申报时间\\
pass\_date&日期&立项时间\\
app\_fund&货币&申报经费\\
pass\_fund&货币&立项经费\\
}{}


表~\ref{projectdatabase}~给出了科研项目数据库表的结构,其中所有类型为变长字符串均的字段均采用MySQL中的“varchar(255)”数据类型,日期采用“date”数据类型,主键id与外键维护人员id采用“int(11)”数据类型,是否国际、是否国家级等类型为布尔的字段均采用“tinyint(1)”型数据类型;申报经费、立项经费不使用字符串类型或浮点类型存储,而是采用整数部分为15位,小数部分为2位的数值类型存储,方便比较和计算,且没有误差,在MySQL对应“decimal(15,2)”数据类型。综合上述各字段数据类型的选择,使用下面的SQL语句,在MySQL数据库中生成科研项目管理模块的数据库表。

\noindent
\ttfamily
\hlstd{CREATE\ TABLE\ IF\ NOT\ EXISTS\ \textasciigrave tbl\textunderscore project\textasciigrave \ (\hspace*{\fill}\\
}\hlstd{\ \ }\hlstd{\textasciigrave id\textasciigrave \ int(}\hlnum{11}\hlstd{)\ NOT\ NULL\ AUTO\textunderscore INCREMENT,\hspace*{\fill}\\
}\hlstd{\ \ }\hlstd{\textasciigrave name\textasciigrave \ varchar(}\hlnum{255}\hlstd{)\ COLLATE\ utf8\textunderscore bin\ DEFAULT\ NULL,\hspace*{\fill}\\
}\hlstd{\ \ }\hlstd{\textasciigrave number\textasciigrave \ varchar(}\hlnum{255}\hlstd{)\ COLLATE\ utf8\textunderscore bin\ DEFAULT\ NULL,\hspace*{\fill}\\
}\hlstd{\ \ }\hlstd{\textasciigrave fund\textunderscore number\textasciigrave \ varchar(}\hlnum{255}\hlstd{)\ COLLATE\ utf8\textunderscore bin\ DEFAULT\ NULL,\hspace*{\fill}\\
}\hlstd{\ \ }\hlstd{\textasciigrave is\textunderscore intl\textasciigrave \ tinyint(}\hlnum{1}\hlstd{)\ DEFAULT\ NULL,\hspace*{\fill}\\
}\hlstd{\ \ }\hlstd{\textasciigrave is\textunderscore national\textasciigrave \ tinyint(}\hlnum{1}\hlstd{)\ DEFAULT\ NULL,\hspace*{\fill}\\
}\hlstd{\ \ }\hlstd{\textasciigrave is\textunderscore provincial\textasciigrave \ tinyint(}\hlnum{1}\hlstd{)\ DEFAULT\ NULL,\hspace*{\fill}\\
}\hlstd{\ \ }\hlstd{\textasciigrave is\textunderscore city\textasciigrave \ tinyint(}\hlnum{1}\hlstd{)\ DEFAULT\ NULL,\hspace*{\fill}\\
}\hlstd{\ \ }\hlstd{\textasciigrave is\textunderscore school\textasciigrave \ tinyint(}\hlnum{1}\hlstd{)\ DEFAULT\ NULL,\hspace*{\fill}\\
}\hlstd{\ \ }\hlstd{\textasciigrave is\textunderscore enterprise\textasciigrave \ tinyint(}\hlnum{1}\hlstd{)\ DEFAULT\ NULL,\hspace*{\fill}\\
}\hlstd{\ \ }\hlstd{\textasciigrave is\textunderscore NSF\textasciigrave \ tinyint(}\hlnum{1}\hlstd{)\ DEFAULT\ NULL,\hspace*{\fill}\\
}\hlstd{\ \ }\hlstd{\textasciigrave is\textunderscore 973\textasciigrave \ tinyint(}\hlnum{1}\hlstd{)\ DEFAULT\ NULL,\hspace*{\fill}\\
}\hlstd{\ \ }\hlstd{\textasciigrave is\textunderscore 863\textasciigrave \ tinyint(}\hlnum{1}\hlstd{)\ DEFAULT\ NULL,\hspace*{\fill}\\
}\hlstd{\ \ }\hlstd{\textasciigrave is\textunderscore NKTRD\textasciigrave \ tinyint(}\hlnum{1}\hlstd{)\ DEFAULT\ NULL,\hspace*{\fill}\\
}\hlstd{\ \ }\hlstd{\textasciigrave is\textunderscore DFME\textasciigrave \ tinyint(}\hlnum{1}\hlstd{)\ DEFAULT\ NULL,\hspace*{\fill}\\
}\hlstd{\ \ }\hlstd{\textasciigrave is\textunderscore major\textasciigrave \ tinyint(}\hlnum{1}\hlstd{)\ DEFAULT\ NULL,\hspace*{\fill}\\
}\hlstd{\ \ }\hlstd{\textasciigrave start\textunderscore date\textasciigrave \ date\ DEFAULT\ NULL,\hspace*{\fill}\\
}\hlstd{\ \ }\hlstd{\textasciigrave deadline\textunderscore date\textasciigrave \ date\ DEFAULT\ NULL,\hspace*{\fill}\\
}\hlstd{\ \ }\hlstd{\textasciigrave conclude\textunderscore date\textasciigrave \ date\ DEFAULT\ NULL,\hspace*{\fill}\\
}\hlstd{\ \ }\hlstd{\textasciigrave app\textunderscore date\textasciigrave \ date\ DEFAULT\ NULL,\hspace*{\fill}\\
}\hlstd{\ \ }\hlstd{\textasciigrave pass\textunderscore date\textasciigrave \ date\ DEFAULT\ NULL,\hspace*{\fill}\\
}\hlstd{\ \ }\hlstd{\textasciigrave app\textunderscore fund\textasciigrave \ decimal(}\hlnum{15}\hlstd{,}\hlnum{2}\hlstd{)\ DEFAULT\ NULL,\hspace*{\fill}\\
}\hlstd{\ \ }\hlstd{\textasciigrave pass\textunderscore fund\textasciigrave \ decimal(}\hlnum{15}\hlstd{,}\hlnum{2}\hlstd{)\ DEFAULT\ NULL,\hspace*{\fill}\\
}\hlstd{\ \ }\hlstd{PRIMARY\ KEY\ (\textasciigrave id\textasciigrave )\hspace*{\fill}\\
)\ ENGINE=InnoDB}\hlstd{\ \ }\hlstd{DEFAULT\ CHARSET=utf8\ COLLATE=utf8\textunderscore bin;}\hspace*{\fill}\\
\mbox{}
\normalfont
\normalsize


\subsubsection{专利}

\threelinetable[H]{peopledatabase}{0.5\textwidth}{lcr}{tbl\_people表的结构}
{字段名称&数据类型&说明\\
}{
id&整形&主键,自动编号\\
name&变长字符串&专利名称\\
app\_date&日期&申请时间\\
app\_number&变长字符串&申请号\\
auth\_date&日期&授权时间\\
auth\_number&变长字符串&授权号\\
is\_intl&布尔&是否国际\\
is\_domestic&布尔&是否国内\\
abstract&变长字符串&专利摘要\\
}{}

表~\ref{peopledatabase}~给出了专利数据库表的结构,其中所有类型为变长字符串均的字段均采用MySQL中的“varchar(255)”数据类型,日期采用“date”数据类型,主键id“int(11)”数据类型,是否国际、是否国家级等类型为布尔的字段均采用“tinyint(1)”型数据类型;申报经费、立项经费不使用字符串类型或浮点类型存储,而是采用整数部分为15位,小数部分为2位的数值类型存储,方便比较和计算,且没有误差,在MySQL对应“decimal(15,2)”数据类型。综合上述各字段数据类型的选择,使用下面的SQL语句,在MySQL数据库中生成专利模块的数据库表。


\noindent
\ttfamily
\hlstd{CREATE\ TABLE\ IF\ NOT\ EXISTS\ \textasciigrave tbl\textunderscore patent\textasciigrave \ (\hspace*{\fill}\\
}\hlstd{\ \ }\hlstd{\textasciigrave id\textasciigrave \ int(}\hlnum{11}\hlstd{)\ NOT\ NULL\ AUTO\textunderscore INCREMENT,\hspace*{\fill}\\
}\hlstd{\ \ }\hlstd{\textasciigrave name\textasciigrave \ varchar(}\hlnum{255}\hlstd{)\ COLLATE\ utf8\textunderscore bin\ NOT\ NULL,\hspace*{\fill}\\
}\hlstd{\ \ }\hlstd{\textasciigrave app\textunderscore date\textasciigrave \ date\ NOT\ NULL,\hspace*{\fill}\\
}\hlstd{\ \ }\hlstd{\textasciigrave app\textunderscore number\textasciigrave \ varchar(}\hlnum{255}\hlstd{)\ COLLATE\ utf8\textunderscore bin\ NOT\ NULL,\hspace*{\fill}\\
}\hlstd{\ \ }\hlstd{\textasciigrave auth\textunderscore number\textasciigrave \ varchar(}\hlnum{255}\hlstd{)\ COLLATE\ utf8\textunderscore bin\ DEFAULT\ NULL,\hspace*{\fill}\\
}\hlstd{\ \ }\hlstd{\textasciigrave auth\textunderscore date\textasciigrave \ date\ DEFAULT\ NULL,\hspace*{\fill}\\
}\hlstd{\ \ }\hlstd{\textasciigrave is\textunderscore intl\textasciigrave \ tinyint(}\hlnum{1}\hlstd{)\ NOT\ NULL,\hspace*{\fill}\\
}\hlstd{\ \ }\hlstd{\textasciigrave is\textunderscore domestic\textasciigrave \ tinyint(}\hlnum{1}\hlstd{)\ NOT\ NULL,\hspace*{\fill}\\
}\hlstd{\ \ }\hlstd{\textasciigrave abstract\textasciigrave \ text\ COLLATE\ utf8\textunderscore bin\ NOT\ NULL,\hspace*{\fill}\\
}\hlstd{\ \ }\hlstd{PRIMARY\ KEY\ (\textasciigrave id\textasciigrave )\hspace*{\fill}\\
)\ ENGINE=InnoDB}\hlstd{\ \ }\hlstd{DEFAULT\ CHARSET=utf8\ COLLATE=utf8\textunderscore bin;}\hspace*{\fill}\\
\mbox{}
\normalfont
\normalsize


\subsubsection{人员}

\threelinetable[H]{peopledatabase}{0.5\textwidth}{lcr}{tbl\_people表的结构}
{字段名称&数据类型&说明\\
}{
id&整形&主键,自动编号\\
name&变长字符串&姓名\\
name\_zh&日期&姓名拼音或英文名\\
type&布尔&教师或者学生\\
description&变长字符串&介绍\\
}{}

% CREATE TABLE IF NOT EXISTS \textasciigrave tbl_people\textasciigrave  (
%   \textasciigrave id\textasciigrave  int(11) NOT NULL AUTO_INCREMENT,
%   \textasciigrave name\textasciigrave  varchar(255) COLLATE utf8_bin NOT NULL,
%   \textasciigrave name\textasciigrave  varchar(255) COLLATE utf8_bin,
%   \textasciigrave type\textasciigrave  tinyint(1),
%   \textasciigrave description\textasciigrave  varchar(255),
%   PRIMARY KEY (\textasciigrave id\textasciigrave )
% ) ENGINE=InnoDB  DEFAULT CHARSET=utf8 COLLATE=utf8_bin;
%
表~\ref{peopledatabase}~给出了人员数据库表的结构,其中所有类型为变长字符串均的字段均采用MySQL中的“varchar(255)”数据类型,主键id“int(11)”数据类型,type字段采用“tinyint(1)”型数据类型;综合上述各字段数据类型的选择,使用下面的SQL语句,在MySQL数据库中生成人员模块的数据库表。

\noindent
\ttfamily
\hlstd{CREATE\ TABLE\ }\hlkwa{IF\ }\hlstd{NOT\ EXISTS\ \textasciigrave tbl\textunderscore people\textasciigrave \ }\hlopt{(}\hspace*{\fill}\\
\hlstd{}\hlstd{\ \ }\hlstd{\textasciigrave id\textasciigrave \ }\hlkwb{int}\hlstd{}\hlopt{(}\hlstd{}\hlnum{11}\hlstd{}\hlopt{)\ }\hlstd{NOT\ NULL\ AUTO\textunderscore INCREMENT}\hlopt{,}\hspace*{\fill}\\
\hlstd{}\hlstd{\ \ }\hlstd{\textasciigrave name\textasciigrave \ }\hlkwd{varchar}\hlstd{}\hlopt{(}\hlstd{}\hlnum{255}\hlstd{}\hlopt{)\ }\hlstd{COLLATE\ utf8\textunderscore bin\ NOT\ NULL}\hlopt{,}\hspace*{\fill}\\
\hlstd{}\hlstd{\ \ }\hlstd{\textasciigrave name\textasciigrave \ }\hlkwd{varchar}\hlstd{}\hlopt{(}\hlstd{}\hlnum{255}\hlstd{}\hlopt{)\ }\hlstd{COLLATE\ utf8\textunderscore bin}\hlopt{,}\hspace*{\fill}\\
\hlstd{}\hlstd{\ \ }\hlstd{\textasciigrave type\textasciigrave \ }\hlkwd{tinyint}\hlstd{}\hlopt{(}\hlstd{}\hlnum{1}\hlstd{}\hlopt{),}\hspace*{\fill}\\
\hlstd{}\hlstd{\ \ }\hlstd{\textasciigrave description\textasciigrave \ }\hlkwd{varchar}\hlstd{}\hlopt{(}\hlstd{}\hlnum{255}\hlstd{}\hlopt{),}\hspace*{\fill}\\
\hlstd{}\hlstd{\ \ }\hlstd{PRIMARY\ }\hlkwd{KEY\ }\hlstd{}\hlopt{(}\hlstd{\textasciigrave id\textasciigrave }\hlopt{)}\hspace*{\fill}\\
\hlstd{}\hlopt{)\ }\hlstd{ENGINE}\hlopt{=}\hlstd{InnoDB}\hlstd{\ \ }\hlstd{}\hlkwa{DEFAULT\ }\hlstd{CHARSET}\hlopt{=}\hlstd{utf8\ COLLATE}\hlopt{=}\hlstd{utf8\textunderscore bin}\hlopt{;}\hlstd{}\hspace*{\fill}\\
\mbox{}
\normalfont
\normalsize

\subsubsection{用户}

\threelinetable[H]{userdatabase}{0.61\textwidth}{lcr}{tbl\_user表的结构}
{字段名称&数据类型&说明\\
}{
id&整形&主键,自动编号\\
username&变长字符串&用户名\\
password&变长字符串&进行散列运算后的密码\\
email&变长字符串&电子邮箱地址\\
description&变长字符串&介绍\\
is\_admin&布尔&是否为超级管理员\\
is\_paper&布尔&是否为学术论文模块管理员\\
is\_project&布尔&是否为科研项目模块管理员\\
is\_patent&布尔&是否为专利模块管理员\\
}{}

表~\ref{userdatabase}~给出了人员数据库表的结构,其中所有类型为变长字符串均的字段均采用MySQL中的“varchar(255)”数据类型,主键id使用“int(11)”数据类型,类型为布尔的字段采用“tinyint(1)”数据类型;综合上述各字段数据类型的选择,使用下面的SQL语句,在MySQL数据库中生成人员模块的数据库表。

% CREATE TABLE IF NOT EXISTS \textasciigrave tbl_user\textasciigrave  (
%   \textasciigrave id\textasciigrave  int(11) NOT NULL AUTO_INCREMENT,
%   \textasciigrave username\textasciigrave  varchar(30) COLLATE utf8_bin NOT NULL,
%   \textasciigrave password\textasciigrave  varchar(255) COLLATE utf8_bin NOT NULL,
%   \textasciigrave email\textasciigrave  varchar(100) COLLATE utf8_bin DEFAULT NULL,
%   \textasciigrave is_admin\textasciigrave  tinyint(1) DEFAULT NULL,
%   \textasciigrave is_paper\textasciigrave  tinyint(1) DEFAULT NULL,
%   \textasciigrave is_project\textasciigrave  tinyint(1) DEFAULT NULL,
%   \textasciigrave is_patent\textasciigrave  tinyint(1) DEFAULT NULL,
%   PRIMARY KEY (\textasciigrave id\textasciigrave )
% ) ENGINE=InnoDB  DEFAULT CHARSET=utf8 COLLATE=utf8_bin;
\noindent
\ttfamily
\hlstd{CREATE\ TABLE\ IF\ NOT\ EXISTS\ \textasciigrave tbl\textunderscore user\textasciigrave \ (\hspace*{\fill}\\
}\hlstd{\ \ }\hlstd{\textasciigrave id\textasciigrave \ int(}\hlnum{11}\hlstd{)\ NOT\ NULL\ AUTO\textunderscore INCREMENT,\hspace*{\fill}\\
}\hlstd{\ \ }\hlstd{\textasciigrave username\textasciigrave \ varchar(}\hlnum{30}\hlstd{)\ COLLATE\ utf8\textunderscore bin\ NOT\ NULL,\hspace*{\fill}\\
}\hlstd{\ \ }\hlstd{\textasciigrave password\textasciigrave \ varchar(}\hlnum{255}\hlstd{)\ COLLATE\ utf8\textunderscore bin\ NOT\ NULL,\hspace*{\fill}\\
}\hlstd{\ \ }\hlstd{\textasciigrave email\textasciigrave \ varchar(}\hlnum{100}\hlstd{)\ COLLATE\ utf8\textunderscore bin\ DEFAULT\ NULL,\hspace*{\fill}\\
}\hlstd{\ \ }\hlstd{\textasciigrave is\textunderscore admin\textasciigrave \ tinyint(}\hlnum{1}\hlstd{)\ DEFAULT\ NULL,\hspace*{\fill}\\
}\hlstd{\ \ }\hlstd{\textasciigrave is\textunderscore paper\textasciigrave \ tinyint(}\hlnum{1}\hlstd{)\ DEFAULT\ NULL,\hspace*{\fill}\\
}\hlstd{\ \ }\hlstd{\textasciigrave is\textunderscore project\textasciigrave \ tinyint(}\hlnum{1}\hlstd{)\ DEFAULT\ NULL,\hspace*{\fill}\\
}\hlstd{\ \ }\hlstd{\textasciigrave is\textunderscore patent\textasciigrave \ tinyint(}\hlnum{1}\hlstd{)\ DEFAULT\ NULL,\hspace*{\fill}\\
}\hlstd{\ \ }\hlstd{PRIMARY\ KEY\ (\textasciigrave id\textasciigrave )\hspace*{\fill}\\
)\ ENGINE=InnoDB}\hlstd{\ \ }\hlstd{DEFAULT\ CHARSET=utf8\ COLLATE=utf8\textunderscore bin;}\hspace*{\fill}\\
\mbox{}
\normalfont
\normalsize

\subsection{数据表关系}
\label{relation}
在科研项目管理中,需要存储和维护不定数目的项目参与人员,科研项目和参与人员之间存在着多对多的关系。在这里显然不能直接在科研项目的数据表中添加若干个字段,存储人员的id;而是要单独建立一个数据表,存储项目和参与人员之间的关系,这样才能满足数据库第二范式,保持数据有效性,节约存储空间。

所谓数据库第二范式(2NF),它要求实体的属性完全依赖于主关键字。所谓完全依赖是指不能存在仅依赖主关键字一部分的属性,如果存在,那么这个属性和主关键字的这一部分应该分离出来形成一个新的实体,新实体与原实体之间是一对多的关系。为实现区分通常需要为表加上一个列,以存储各个实例的惟一标识。简而言之,第二范式就是属性完全依赖于主键。

表~\ref{projectpeopledb}~给出了存储科研项目与实际执行人员的关系的数据表结构。这样在查找某一个科研项目记录的时候,只需要在存储科研项目与实际执行人员的关系的数据表中找到满足科研项目id等于当前科研项目记录id,就可以得到该项目的所有实际执行人员id(可能有多个),再依次地按照这些际执行人员id在人员数据表中查找,便可以得到某一个科研项目的所有实际执行人员。

\threelinetable[H]{projectpeopledb}{0.6\textwidth}{lcr}{tbl\_project\_people\_execute表的结构}
{字段名称&数据类型&说明\\
}{
project\_id&整形&科研项目id,外键,不能为空\\
people\_id&整形&人员id,外键,不能为空\\
seq&整形&人员在单个项目中的顺序\\
}{}


使用下面的SQL语句生成存储科研项目与实际执行人员的关系的数据表:

\noindent
\ttfamily
\hlstd{CREATE\ TABLE\ IF\ NOT\ EXISTS\ \textasciigrave tbl\textunderscore project\textunderscore people\textunderscore execute\textasciigrave \ (\hspace*{\fill}\\
}\hlstd{\ \ }\hlstd{\textasciigrave project\textunderscore id\textasciigrave \ int(}\hlnum{11}\hlstd{)\ NOT\ NULL,\hspace*{\fill}\\
}\hlstd{\ \ }\hlstd{\textasciigrave people\textunderscore id\textasciigrave \ int(}\hlnum{11}\hlstd{)\ NOT\ NULL,\hspace*{\fill}\\
}\hlstd{\ \ }\hlstd{\textasciigrave seq\textasciigrave \ int(}\hlnum{11}\hlstd{)\ NOT\ NULL,\hspace*{\fill}\\
}\hlstd{\ \ }\hlstd{PRIMARY\ KEY\ (\textasciigrave project\textunderscore id\textasciigrave ,\textasciigrave people\textunderscore id\textasciigrave ),\hspace*{\fill}\\
)\ ENGINE=InnoDB\ DEFAULT\ CHARSET=utf8\ COLLATE=utf8\textunderscore bin;}\hspace*{\fill}\\
\mbox{}
\normalfont
\normalsize

使用下面的SQL语句给该数据表添加外键限制,以增强本系统数据的有效性和一致性:

\noindent
\ttfamily
\hlstd{ALTER\ TABLE\ \textasciigrave tbl\textunderscore project\textunderscore people\textunderscore execute\textasciigrave \hspace*{\fill}\\
}\hlstd{\ \ }\hlstd{ADD\ CONSTRAINT\ \textasciigrave tbl\textunderscore project\textunderscore people\textunderscore execute\textunderscore ibfk\textunderscore 1\textasciigrave \ \hspace*{\fill}\\
}\hlstd{\ \ \ \ }\hlstd{FOREIGN\ KEY\ (\textasciigrave project\textunderscore id\textasciigrave )\hspace*{\fill}\\
}\hlstd{\ \ \ \ }\hlstd{REFERENCES\ \textasciigrave tbl\textunderscore project\textasciigrave \ (\textasciigrave id\textasciigrave )\ ON\ DELETE\ CASCADE\ ON\ \Righttorque\hspace*{\fill}\\
}\hlstd{\ \ \ \ }\hlstd{UPDATE\ CASCADE,\hspace*{\fill}\\
}\hlstd{\ \ }\hlstd{ADD\ CONSTRAINT\ \textasciigrave tbl\textunderscore project\textunderscore people\textunderscore execute\textunderscore ibfk\textunderscore 2\textasciigrave \hspace*{\fill}\\
}\hlstd{\ \ \ \ }\hlstd{FOREIGN\ KEY\ (\textasciigrave people\textunderscore id\textasciigrave )\hspace*{\fill}\\
}\hlstd{\ \ \ \ }\hlstd{REFERENCES\ \textasciigrave tbl\textunderscore people\textasciigrave \ (\textasciigrave id\textasciigrave )\ ON\ DELETE\ CASCADE\ ON\ \Righttorque\hspace*{\fill}\\
}\hlstd{\ \ \ \ }\hlstd{UPDATE\ CASCADE;}\hspace*{\fill}\\
\mbox{}
\normalfont
\normalsize

通过使用SQL的“JOIN”命令,可以从存储科研项目与实际执行人员的关系的数据表中得到某个项目的实际执行人员。例如,下面的SQL查询得到id为1的科研项目实际执行人员:

\noindent
\ttfamily
\hlstd{SELECT\ \textasciigrave execute\textunderscore \textasciigrave .\textasciigrave id\textasciigrave \ AS\ \textasciigrave t1\textunderscore c0\textasciigrave ,\ \textasciigrave execute\textunderscore \textasciigrave .\textasciigrave name\textasciigrave \ AS\hspace*{\fill}\\
\textasciigrave t1\textunderscore c1\textasciigrave \ FROM\ \textasciigrave tbl\textunderscore people\textasciigrave \ \textasciigrave execute\textunderscore \textasciigrave }\hlstd{\ \ }\hlstd{INNER\ JOIN\hspace*{\fill}\\
\textasciigrave tbl\textunderscore project\textunderscore people\textunderscore execute\textasciigrave \ \textasciigrave execute\textunderscore peoples\textunderscore execute\textunderscore \textasciigrave \ ON\hspace*{\fill}\\
(\textasciigrave execute\textunderscore peoples\textunderscore execute\textunderscore \textasciigrave .\textasciigrave project\textunderscore id\textasciigrave =}\hlnum{1}\hlstd{)\ AND\hspace*{\fill}\\
(\textasciigrave execute\textunderscore \textasciigrave .\textasciigrave id\textasciigrave =\textasciigrave execute\textunderscore peoples\textunderscore execute\textunderscore \textasciigrave .\textasciigrave people\textunderscore id\textasciigrave )\ \Righttorque\hspace*{\fill}\\
ORDER\ BY\hspace*{\fill}\\
execute\textunderscore peoples\textunderscore execute\textunderscore .seq;}\hspace*{\fill}\\
\mbox{}
\normalfont
\normalsize
% SELECT \textasciigrave execute_\textasciigrave .\textasciigrave id\textasciigrave  AS \textasciigrave t1_c0\textasciigrave , \textasciigrave execute_\textasciigrave .\textasciigrave name\textasciigrave  AS
% \textasciigrave t1_c1\textasciigrave  FROM \textasciigrave tbl_people\textasciigrave  \textasciigrave execute_\textasciigrave   INNER JOIN
% \textasciigrave tbl_project_people_execute\textasciigrave  \textasciigrave execute_peoples_execute_\textasciigrave  ON
% (\textasciigrave execute_peoples_execute_\textasciigrave .\textasciigrave project_id\textasciigrave =1) AND
% (\textasciigrave execute_\textasciigrave .\textasciigrave id\textasciigrave =\textasciigrave execute_peoples_execute_\textasciigrave .\textasciigrave people_id\textasciigrave ) ORDER BY
% execute_peoples_execute_.seq;

如图~\ref{pngprojecter.png}~所示,这是与科研项目有关的实体-关系图(ER图),图中展示了科研项目与实际执行人员的关系、科研项目与责任书人员的关系、学术论文与资助科研项目的关系以及学术论文与报账科研项目之间的关系。

\pic[H]{与科研项目有关的实体-关系图}{width=1.0\textwidth}{pngprojecter.png}

同样地,在本系统中,还需要分别存储和维护科研项目与责任书人员的关系、学术论文与资助科研项目的关系、学术论文与报账科研项目之间的关系、科研项目与维护人员的关系、学术论文与作者的关系、学术论文与维护人员的关系、发明和发明人的关系,皆采用类似上述建立一个关系数据库表的方式来实现,限于篇幅关系,在这里不再赘述。

\section{MVC开发规范}
在~Yii~框架中,偏爱规范胜于配置。遵循规范可使你能够创建成熟的Yii应用而不需要编写、维护复杂的配置。 

下面我们讲解 Yii 编程中推荐的开发规范。 为简单起见,我们假设 WebRoot 是 Yii 应用安装的目录。

\subsection{URL}
默认情况下,Yii 识别如下格式的 URL:

http://hostname/index.php?r=ControllerID/ActionID

其中,~r~是一个GET的查询参数,意为路由(route) ,它可以被Yii解析为 控制器和动作。 如果 ActionID 被省略,控制器将使用默认的动作(在CController::defaultAction中定义); 如果 ControllerID 也被省略(或者 r 变量不存在),应用将使用默认的控制器 (在CWebApplication::defaultController中定义)。

\subsection{文件}

命名和使用文件的规范取决于它们的类型。

类文件应以它们包含的公有类命名。例如, CController 类位于 CController.php 文件中。 公有类是可以被任何其他类使用的类。每个类文件应包含最多一个公有类。 私有类(只能被一个公有类使用的类)可以放在使用此类的公有类所在的文件中。

视图文件应以视图的名字命名。例如, index 视图位于 index.php 文件中。 视图文件是一个PHP脚本文件,它包含了用于呈现内容的 HTML和PHP代码。

配置文件可以任意命名。 配置文件是一个PHP脚本,它的主要目的是返回一个体现配置的关联数组。

\subsection{目录}
\label{yiiprotocoldir}

Yii 假定了一系列默认的目录用于不同的场合:
\begin{itemize}
\item WebRoot/protected: 这是 应用基础目录, 是放置所有安全敏感的PHP脚本和数据文件的地方。Yii 有一个默认的 application 别名指向此目录。 此目录及目录中的文件应该保护起来防止Web用户访问。
\item WebRoot/protected/models:此目录放置所有的模型文件。
\item WebRoot/protected/controllers: 此目录放置所有控制器类文件。
\item WebRoot/protected/views: 此目录放置所有试图文件, 包含控制器视图,布局视图和系统视图。 它可以通过
\item WebRoot/protected/views/ControllerID: 此目录放置单个控制器类中使用的视图文件。 此处的 ControllerID 是指控制器的 ID 。
\end{itemize}

\section{模型层实现}

在本小节中,仅给出了数据表项最多,数据关系最复杂的科研项目模型的实现,人员模型、学术论文模型、专利模型的实现方法与科研项目模型相类似,在这里不再赘述。

Yii框架提供了易用的AR(Active Record)实现,AR 是一个流行的 对象-关系映射 (ORM) 技术。 每个 AR 类代表一个数据表(或视图),数据表(或视图)的列在 AR 类中体现为类的属性,一个 AR 实例则表示表中的一行。 常见的 CRUD 操作作为 AR 的方法实现。在本系统中,使用AR技术来实现各个模型。

按照第~\ref{yiiprotocoldir}~节中所提高到的文件与目录规范,应该将本小节中所提到的科研项目模型类的实现,放在~WebRoot/protected/models/~目录中的~Project.php~文件中。

\subsection{建立数据库连接}

AR 依靠一个数据库连接以执行数据库相关的操作。在Yii框架目录下的config.php中按照下面的PHP代码配置数据库链接。其中,host是MySQL服务器的地址,在开发服务器上,即为本机,填写本地环回地址;dbname为MySQL中存储本系统的数据库名称,根据需要填写,在开发服务器上填写为testdrive;username和password分别为mysql的密码;charset填写在第~\pageref{utf8}页~第~\ref{utf8}~节中选择的utf-8。

% return array(
% 	//其它配置
%     'db'=>array(
% 			'connectionString' => 'mysql:host=127.0.0.1;dbname=testdrive',
% 			'emulatePrepare' => true,
% 			'username' => 'root',
% 			'password' => 'test',
% 			'charset' => 'utf8',
% 		),
% );

\noindent
\ttfamily
\hlstd{}\hlkwa{return\ array}\hlstd{}\hlopt{(}\hspace*{\fill}\\
\hlstd{}\hlstd{\ \ \ \ }\hlstd{}\hlslc{//其它配置}\hspace*{\fill}\\
\hlstd{}\hlstd{\ \ \ \ }\hlstd{}\hlstr{'db'}\hlstd{}\hlopt{=$>$}\hlstd{}\hlkwa{array}\hlstd{}\hlopt{(}\hspace*{\fill}\\
\hlstd{}\hlstd{\ \ \ \ \ \ \ \ \ \ \ \ }\hlstd{}\hlstr{'connectionString'}\hlstd{\ }\hlopt{=$>$\ }\hlstd{}\hlstr{'mysql:host=127.0.0.1;}\Righttorque\hspace*{\fill}\\
\hlstr{}\hlstd{\ \ \ \ \ \ \ \ \ \ \ \ }\hlstr{dbname=testdrive'}\hlstd{}\hlopt{,}\hspace*{\fill}\\
\hlstd{}\hlstd{\ \ \ \ \ \ \ \ \ \ \ \ }\hlstd{}\hlstr{'emulatePrepare'}\hlstd{\ }\hlopt{=$>$\ }\hlstd{true}\hlopt{,}\hspace*{\fill}\\
\hlstd{}\hlstd{\ \ \ \ \ \ \ \ \ \ \ \ }\hlstd{}\hlstr{'username'}\hlstd{\ }\hlopt{=$>$\ }\hlstd{}\hlstr{'root'}\hlstd{}\hlopt{,}\hspace*{\fill}\\
\hlstd{}\hlstd{\ \ \ \ \ \ \ \ \ \ \ \ }\hlstd{}\hlstr{'password'}\hlstd{\ }\hlopt{=$>$\ }\hlstd{}\hlstr{'test'}\hlstd{}\hlopt{,}\hspace*{\fill}\\
\hlstd{}\hlstd{\ \ \ \ \ \ \ \ \ \ \ \ }\hlstd{}\hlstr{'charset'}\hlstd{\ }\hlopt{=$>$\ }\hlstd{}\hlstr{'utf8'}\hlstd{}\hlopt{,}\hspace*{\fill}\\
\hlstd{}\hlstd{\ \ \ \ \ \ \ \ }\hlstd{}\hlopt{),}\hspace*{\fill}\\
\hlstd{}\hlopt{);}\hlstd{}\hspace*{\fill}\\
\mbox{}
\normalfont
\normalsize

\subsection{定义AR类}

Yii~框架提供了一个名为~CActiveRecord~的~AR~类,如下面的~PHP~代码所示:让科研项目模型类~Project~继承自~CActiveRecord~类,并复写父类的~tableName()~方法,使它返回我们存储科研项目的数据库表的名称,即“tbl\_project”,便可以使用~CActiveRecord~提供的接口以面向对象的方式对科研项目的数据库表进行增加、删除、修改、查询了。

\noindent
\ttfamily
\hlstd{}\hlkwa{class\ }\hlstd{Project\ }\hlkwa{extends\ }\hlstd{CActiveRecord}\hspace*{\fill}\\
\hlopt{\{}\hspace*{\fill}\\
\hlstd{}\hlstd{\ \ \ \ }\hlstd{}\hlkwa{public\ function\ }\hlstd{}\hlkwd{tableName}\hlstd{}\hlopt{()}\hspace*{\fill}\\
\hlstd{}\hlstd{\ \ \ \ }\hlstd{}\hlopt{\{}\hspace*{\fill}\\
\hlstd{}\hlstd{\ \ \ \ \ \ \ \ }\hlstd{}\hlkwa{return\ }\hlstd{}\hlstr{'tbl\textunderscore project'}\hlstd{}\hlopt{;}\hspace*{\fill}\\
\hlstd{}\hlstd{\ \ \ \ }\hlstd{}\hlopt{\}}\hspace*{\fill}\\
\hlstd{}}\hspace*{\fill}\\
\mbox{}
\normalfont
\normalsize
%
% class Project extends CActiveRecord
% {
% 	public function tableName()
% 	{
% 		return 'tbl_project';
% 	}
% }
%
%
% $project=new Project;
% $project->name='项目名称';
% //对其他字段进行赋值
% $project->save();
%
% // 查找满足指定条件的结果中的第一行
% $project=Project::model()->find($condition,$params);
% // 查找具有指定主键值的那一行
% $project=Project::model()->findByPk($projectID,$condition,$params);
% // 查找具有指定属性值的行
% $project=Project::model()->findByAttributes($attributes,$condition,$params);
% // 通过指定的 SQL 语句查找结果中的第一行
% $project=Project::model()->findBySql($sql,$params);
%
% $projects=Project::model()->findAll($condition,$params);
% // 查找带有指定主键的所有行
% $projects=Project::model()->findAllByPk($projectIDs,$condition,$params);
% // 查找带有指定属性值的所有行
% $projects=Project::model()->findAllByAttributes($attributes,$condition,$params);
% // 通过指定的SQL语句查找所有行
% $projects=Project::model()->findAllBySql($sql,$params);
%
% $project=Project::model()->findByPk(10);
% $project->name='新的项目名称';
% $project->save(); // 将更改保存到数据库
%
% $project=Project::model()->findByPk(10); // 假设有一个科研项目,其 ID 为 10
% $project->delete(); // 从数据表中删除此行
%
% // 删除符合指定条件的行
% Project::model()->deleteAll($condition,$params);
% // 删除符合指定条件和主键的行
% Project::model()->deleteByPk($pk,$condition,$params);

\subsection{实现增删操作}
增删改查操作指对数据库表记录的增加、删除、修改和查询,在本系统的控制器层中,可以使用面向对象的方式对科研项目的数据库表进行操作了:
\begin{enumerate}
\item 增加新的科研项目的条目:\\
\noindent
\ttfamily
\hlstd{}\hlkwc{\$project}\hlstd{}\hlopt{=}\hlstd{}\hlkwa{new\ }\hlstd{Project}\hlopt{;}\hspace*{\fill}\\
\hlstd{}\hlkwc{\$project}\hlstd{}\hlopt{{-}$>$}\hlstd{name}\hlopt{=}\hlstd{}\hlstr{'项目名称'}\hlstd{}\hlopt{;}\hspace*{\fill}\\
\hlstd{}\hlslc{//对其他字段进行赋值}\hspace*{\fill}\\
\hlstd{}\hlkwc{\$project}\hlstd{}\hlopt{{-}$>$}\hlstd{}\hlkwd{save}\hlstd{}\hlopt{();}\hlstd{}\hspace*{\fill}\\
\mbox{}
\normalfont
\normalsize
\item 对现有科研项目的条目进行查询:\\
\noindent
\ttfamily
\hlstd{}\hlslc{//\ 查找满足指定条件的结果中的第一行}\hspace*{\fill}\\
\hlstd{}\hlkwc{\$project}\hlstd{}\hlopt{=}\hlstd{Project}\hlopt{::}\hlstd{}\hlkwd{model}\hlstd{}\hlopt{(){-}$>$}\hlstd{}\hlkwd{find}\hlstd{}\hlopt{(}\hlstd{}\hlkwc{\$condition}\hlstd{}\hlopt{,}\hlstd{}\hlkwc{\$params}\hlstd{}\hlopt{);}\hspace*{\fill}\\
\hlstd{}\hlslc{//\ 查找具有指定主键值的那一行}\hspace*{\fill}\\
\hlstd{}\hlkwc{\$project}\hlstd{}\hlopt{=}\hlstd{Project}\hlopt{::}\hlstd{}\hlkwd{model}\hlstd{}\hlopt{(){-}$>$}\hlstd{}\hlkwd{findByPk}\hlstd{}\hlopt{(}\hlstd{}\hlkwc{\$projectID}\hlstd{}\hlopt{,}\hlstd{}\hlkwc{\$condition}\hlstd{}\hlopt{,}\Righttorque\hspace*{\fill}\\
\hlstd{}\hlkwc{\$params}\hlstd{}\hlopt{);}\hspace*{\fill}\\
\hlstd{}\hlslc{//\ 查找具有指定属性值的行}\hspace*{\fill}\\
\hlstd{}\hlkwc{\$project}\hlstd{}\hlopt{=}\hlstd{Project}\hlopt{::}\hlstd{}\hlkwd{model}\hlstd{}\hlopt{(){-}$>$}\hlstd{}\hlkwd{findByAttributes}\hlstd{}\hlopt{(}\hlstd{}\hlkwc{\$attributes}\hlstd{}\hlopt{,}\Righttorque\hspace*{\fill}\\
\hlstd{}\hlkwc{\$condition}\hlstd{}\hlopt{,}\hlstd{}\hlkwc{\$params}\hlstd{}\hlopt{);}\hspace*{\fill}\\
\hlstd{}\hlslc{//\ 通过指定的\ SQL\ 语句查找结果中的第一行}\hspace*{\fill}\\
\hlstd{}\hlkwc{\$project}\hlstd{}\hlopt{=}\hlstd{Project}\hlopt{::}\hlstd{}\hlkwd{model}\hlstd{}\hlopt{(){-}$>$}\hlstd{}\hlkwd{findBySql}\hlstd{}\hlopt{(}\hlstd{}\hlkwc{\$sql}\hlstd{}\hlopt{,}\hlstd{}\hlkwc{\$params}\hlstd{}\hlopt{);}\hspace*{\fill}\\
\hlstd{}\hspace*{\fill}\\
\hlkwc{\$projects}\hlstd{}\hlopt{=}\hlstd{Project}\hlopt{::}\hlstd{}\hlkwd{model}\hlstd{}\hlopt{(){-}$>$}\hlstd{}\hlkwd{findAll}\hlstd{}\hlopt{(}\hlstd{}\hlkwc{\$condition}\hlstd{}\hlopt{,}\hlstd{}\hlkwc{\$params}\hlstd{}\hlopt{);}\hspace*{\fill}\\
\hlstd{}\hlslc{//\ 查找带有指定主键的所有行}\hspace*{\fill}\\
\hlstd{}\hlkwc{\$projects}\hlstd{}\hlopt{=}\hlstd{Project}\hlopt{::}\hlstd{}\hlkwd{model}\hlstd{}\hlopt{(){-}$>$}\hlstd{}\hlkwd{findAllByPk}\hlstd{}\hlopt{(}\hlstd{}\hlkwc{\$projectIDs}\hlstd{}\hlopt{,}\Righttorque\hspace*{\fill}\\
\hlstd{}\hlkwc{\$condition}\hlstd{}\hlopt{,}\hlstd{}\hlkwc{\$params}\hlstd{}\hlopt{);}\hspace*{\fill}\\
\hlstd{}\hlslc{//\ 查找带有指定属性值的所有行}\hspace*{\fill}\\
\hlstd{}\hlkwc{\$projects}\hlstd{}\hlopt{=}\hlstd{Project}\hlopt{::}\hlstd{}\hlkwd{model}\hlstd{}\hlopt{(){-}$>$}\hlstd{}\hlkwd{findAllByAttributes}\hlstd{}\hlopt{(}\hlstd{}\hlkwc{\$attributes}\hlstd{}\hlopt{,}\Righttorque\hspace*{\fill}\\
\hlstd{}\hlkwc{\$condition}\hlstd{}\hlopt{,}\hlstd{}\hlkwc{\$params}\hlstd{}\hlopt{);}\hspace*{\fill}\\
\hlstd{}\hlslc{//\ 通过指定的SQL语句查找所有行}\hspace*{\fill}\\
\hlstd{}\hlkwc{\$projects}\hlstd{}\hlopt{=}\hlstd{Project}\hlopt{::}\hlstd{}\hlkwd{model}\hlstd{}\hlopt{(){-}$>$}\hlstd{}\hlkwd{findAllBySql}\hlstd{}\hlopt{(}\hlstd{}\hlkwc{\$sql}\hlstd{}\hlopt{,}\hlstd{}\hlkwc{\$params}\hlstd{}\hlopt{);}\hlstd{}\hspace*{\fill}\\
\mbox{}
\normalfont
\normalsize

\item 对现有科研项目的条目进行修改:\\
\noindent
\ttfamily
\hlstd{}\hlkwc{\$project}\hlstd{}\hlopt{=}\hlstd{Project}\hlopt{::}\hlstd{}\hlkwd{model}\hlstd{}\hlopt{(){-}$>$}\hlstd{}\hlkwd{findByPk}\hlstd{}\hlopt{(}\hlstd{}\hlnum{10}\hlstd{}\hlopt{);}\hspace*{\fill}\\
\hlstd{}\hlkwc{\$project}\hlstd{}\hlopt{{-}$>$}\hlstd{name}\hlopt{=}\hlstd{}\hlstr{'新的项目名称'}\hlstd{}\hlopt{;}\hspace*{\fill}\\
\hlstd{}\hlkwc{\$project}\hlstd{}\hlopt{{-}$>$}\hlstd{}\hlkwd{save}\hlstd{}\hlopt{();\ }\hlstd{}\hlslc{//\ 将更改保存到数据库}\hlstd{}\hspace*{\fill}\\
\mbox{}
\normalfont
\normalsize

\item 对现有科研项目的条目进行删除:\\
\noindent
\ttfamily
\hlstd{}\hlkwc{\$project}\hlstd{}\hlopt{=}\hlstd{Project}\hlopt{::}\hlstd{}\hlkwd{model}\hlstd{}\hlopt{(){-}$>$}\hlstd{}\hlkwd{findByPk}\hlstd{}\hlopt{(}\hlstd{}\hlnum{10}\hlstd{}\hlopt{);\ }\hlstd{}\hlslc{//\ }\Righttorque\hspace*{\fill}\\
\hlslc{假设有一个科研项目,其\ ID\ 为\ 10}\hspace*{\fill}\\
\hlstd{}\hlkwc{\$project}\hlstd{}\hlopt{{-}$>$}\hlstd{}\hlkwd{delete}\hlstd{}\hlopt{();\ }\hlstd{}\hlslc{//\ 从数据表中删除此行}\hspace*{\fill}\\
\hlstd{}\hspace*{\fill}\\
\hlslc{//\ 删除符合指定条件的行}\hspace*{\fill}\\
\hlstd{Project}\hlopt{::}\hlstd{}\hlkwd{model}\hlstd{}\hlopt{(){-}$>$}\hlstd{}\hlkwd{deleteAll}\hlstd{}\hlopt{(}\hlstd{}\hlkwc{\$condition}\hlstd{}\hlopt{,}\hlstd{}\hlkwc{\$params}\hlstd{}\hlopt{);}\hspace*{\fill}\\
\hlstd{}\hlslc{//\ 删除符合指定条件和主键的行}\hspace*{\fill}\\
\hlstd{Project}\hlopt{::}\hlstd{}\hlkwd{model}\hlstd{}\hlopt{(){-}$>$}\hlstd{}\hlkwd{deleteByPk}\hlstd{}\hlopt{(}\hlstd{}\hlkwc{\$pk}\hlstd{}\hlopt{,}\hlstd{}\hlkwc{\$condition}\hlstd{}\hlopt{,}\hlstd{}\hlkwc{\$params}\hlstd{}\hlopt{);}\hlstd{}\hspace*{\fill}\\
\mbox{}
\normalfont
\normalsize

\end{enumerate}

\subsection{定义数据库表关系}
在上一节中,已经实现了使用 Active Record (AR) 从单个数据表对条目进行增删改查的操作。 在本节中,将使用 AR 连接多个相关数据表并取回关联(对应MySQL中的“JOIN”语句)后的数据表。

在使用 AR 执行关联查询之前,需要让 AR 知道一个 AR 类是怎样关联到另一个的。

两个 AR 类之间的关系直接通过 AR 类所代表的数据表之间的关系相关联。在 AR 中,定义了三种关系类型:
\begin{enumerate}
\item BELONGS\_TO:一对多
\item HAS\_MANY:多对一
\item MANY\_MANY:多对多
\end{enumerate}

在本系统的科研项目管理模块中,科研项目与维护者之间的关系是多对一,而科研项目与执行人员之间的关系是多对多。

如下面的PHP代码所示,通过在Project类中复写父类~AR~中定义关系的~relations()~方法,可以定义科研项目与其它数据库表的关系。

% public function relations()
% {
% 	return array(
% 		'liability_peoples' => array(
% 			self::MANY_MANY, 
% 			'People', 
% 			'tbl_project_people_liability(project_id, people_id)',
% 			'order'=>'liability_peoples_liability_.seq',
% 			'alias'=>'liability_'
% 		),
%         'execute_peoples' => array(
%         	self::MANY_MANY, 
%         	'People', 
%         	'tbl_project_people_execute(project_id, people_id)',
%         	'order'=>'execute_peoples_execute_.seq',
%         	'alias'=>'execute_'
%         ),
% 	);

% }
\noindent
\ttfamily
\hlstd{}\hlkwa{public\ function\ }\hlstd{}\hlkwd{relations}\hlstd{}\hlopt{()}\hspace*{\fill}\\
\hlstd{}\hlopt{\{}\hspace*{\fill}\\
\hlstd{}\hlstd{\ \ \ \ }\hlstd{}\hlkwa{return\ array}\hlstd{}\hlopt{(}\hspace*{\fill}\\
\hlstd{}\hlstd{\ \ \ \ \ \ \ \ }\hlstd{}\hlstr{'liability\textunderscore peoples'}\hlstd{\ }\hlopt{=$>$\ }\hlstd{}\hlkwa{array}\hlstd{}\hlopt{(}\hspace*{\fill}\\
\hlstd{}\hlstd{\ \ \ \ \ \ \ \ \ \ \ \ }\hlstd{self}\hlopt{::}\hlstd{MANY\textunderscore MANY}\hlopt{,\ }\hspace*{\fill}\\
\hlstd{}\hlstd{\ \ \ \ \ \ \ \ \ \ \ \ }\hlstd{}\hlstr{'People'}\hlstd{}\hlopt{,\ }\hspace*{\fill}\\
\hlstd{}\hlstd{\ \ \ \ \ \ \ \ \ \ \ \ }\hlstd{}\hlstr{'tbl\textunderscore project\textunderscore people\textunderscore liability(project\textunderscore id,\ }\Righttorque\hspace*{\fill}\\
\hlstr{}\hlstd{\ \ \ \ \ \ \ \ \ \ \ \ }\hlstr{people\textunderscore id)'}\hlstd{}\hlopt{,}\hspace*{\fill}\\
\hlstd{}\hlstd{\ \ \ \ \ \ \ \ \ \ \ \ }\hlstd{}\hlstr{'order'}\hlstd{}\hlopt{=$>$}\hlstd{}\hlstr{'liability\textunderscore peoples\textunderscore liability\textunderscore .seq'}\hlstd{}\hlopt{,}\hspace*{\fill}\\
\hlstd{}\hlstd{\ \ \ \ \ \ \ \ \ \ \ \ }\hlstd{}\hlstr{'alias'}\hlstd{}\hlopt{=$>$}\hlstd{}\hlstr{'liability\textunderscore '}\hlstd{\hspace*{\fill}\\
}\hlstd{\ \ \ \ \ \ \ \ }\hlstd{}\hlopt{),}\hspace*{\fill}\\
\hlstd{}\hlstd{\ \ \ \ \ \ \ \ }\hlstd{}\hlstr{'execute\textunderscore peoples'}\hlstd{\ }\hlopt{=$>$\ }\hlstd{}\hlkwa{array}\hlstd{}\hlopt{(}\hspace*{\fill}\\
\hlstd{}\hlstd{\ \ \ \ \ \ \ \ \ \ \ \ }\hlstd{self}\hlopt{::}\hlstd{MANY\textunderscore MANY}\hlopt{,\ }\hspace*{\fill}\\
\hlstd{}\hlstd{\ \ \ \ \ \ \ \ \ \ \ \ }\hlstd{}\hlstr{'People'}\hlstd{}\hlopt{,\ }\hspace*{\fill}\\
\hlstd{}\hlstd{\ \ \ \ \ \ \ \ \ \ \ \ }\hlstd{}\hlstr{'tbl\textunderscore project\textunderscore people\textunderscore execute(project\textunderscore id,\ }\Righttorque\hspace*{\fill}\\
\hlstr{}\hlstd{\ \ \ \ \ \ \ \ \ \ \ \ }\hlstr{people\textunderscore id)'}\hlstd{}\hlopt{,}\hspace*{\fill}\\
\hlstd{}\hlstd{\ \ \ \ \ \ \ \ \ \ \ \ }\hlstd{}\hlstr{'order'}\hlstd{}\hlopt{=$>$}\hlstd{}\hlstr{'execute\textunderscore peoples\textunderscore execute\textunderscore .seq'}\hlstd{}\hlopt{,}\hspace*{\fill}\\
\hlstd{}\hlstd{\ \ \ \ \ \ \ \ \ \ \ \ }\hlstd{}\hlstr{'alias'}\hlstd{}\hlopt{=$>$}\hlstd{}\hlstr{'execute\textunderscore '}\hlstd{\hspace*{\fill}\\
}\hlstd{\ \ \ \ \ \ \ \ }\hlstd{}\hlopt{),}\hspace*{\fill}\\
\hlstd{}\hlstd{\ \ \ \ }\hlstd{}\hlopt{);}\hspace*{\fill}\\
\hlstd{}\hspace*{\fill}\\
\hlopt{\}}\hlstd{}\hspace*{\fill}\\
\mbox{}
\normalfont
\normalsize

在定义了数据库表关系之后,可以在控制器层中方便地执行关联查询以及增删改查,就像访问科研项目的模型类Project本身的属性一样:

% //获取 ID 为 10 的科研项目
% $project=Project::model()->findByPk(10);
% //获取此科研项目的维护者的姓名: 此处将执行一个关联查询。
% $maintainter=$project->maintainter->name;
% //获取此科研项目的所有实际执行人员
% //将返回一个人员对象的数组
% $exec_peoples=$project->execute_peoples;
\noindent
\ttfamily
\hlstd{}\hlslc{//获取\ ID\ 为\ 10\ 的科研项目}\hspace*{\fill}\\
\hlstd{}\hlkwc{\$project}\hlstd{}\hlopt{=}\hlstd{Project}\hlopt{::}\hlstd{}\hlkwd{model}\hlstd{}\hlopt{(){-}$>$}\hlstd{}\hlkwd{findByPk}\hlstd{}\hlopt{(}\hlstd{}\hlnum{10}\hlstd{}\hlopt{);}\hspace*{\fill}\\
\hlstd{}\hlslc{//获取此科研项目的维护者的姓名:\ }\Righttorque\hspace*{\fill}\\
\hlslc{此处将执行一个关联查询。}\hspace*{\fill}\\
\hlstd{}\hlkwc{\$maintainter}\hlstd{}\hlopt{=}\hlstd{}\hlkwc{\$project}\hlstd{}\hlopt{{-}$>$}\hlstd{maintainter}\hlopt{{-}$>$}\hlstd{name}\hlopt{;}\hspace*{\fill}\\
\hlstd{}\hlslc{//获取此科研项目的所有实际执行人员}\hspace*{\fill}\\
\hlstd{}\hlslc{//将返回一个人员对象的数组}\hspace*{\fill}\\
\hlstd{}\hlkwc{\$exec\textunderscore peoples}\hlstd{}\hlopt{=}\hlstd{}\hlkwc{\$project}\hlstd{}\hlopt{{-}$>$}\hlstd{execute\textunderscore peoples}\hlopt{;}\hlstd{}\hspace*{\fill}\\
\mbox{}
\normalfont
\normalsize

\subsection{验证数据有效性}

% public function rules()
% {
% 	return array(
% 		array('name','required'),
% 		array('is_intl, is_national, is_provincial, is_city, is_school, is_enterprise, is_NSF, is_973, is_863, is_NKTRD, is_DFME, is_major', 'numerical', 'integerOnly'=>true),
% 		array('name, number, fund_number', 'length', 'max'=>255),
% 		array('app_fund, pass_fund', 'length', 'max'=>15),
% 		array('start_date, deadline_date, conclude_date, app_date, pass_date', 'safe'),
% 		array('id, name, number, fund_number, is_intl, is_national, is_provincial, is_city, is_school, is_enterprise, is_NSF, is_973, is_863, is_NKTRD, is_DFME, is_major, start_date, deadline_date, conclude_date, app_date, pass_date, app_fund, pass_fund, ', 'safe', 'on'=>'search'),
% 	);
% }


当插入或更新一行时,我们常常需要检查列的值是否符合相应的规则。 如果列的值是由最终用户提供的,这一点就更加重要。总体来说,为了防止恶意攻击,永远不能相信任何来自客户端的数据。为了维护本系统数据的有效性,在本系统的模型层需要定义验证输入数据有效性的规则。

当调用~AR~类的实例的save()方法时,AR~类的实例会自动执行数据验证。验证是基于在~AR~类的rules() 方法中指定的规则进行的。如下面给出的PHP代码,通过根据在第~\pageref{project}~页第~\ref{project}~节建立的数据库表的结构,在科研项目模型类Project中复写父类的rules()方法,可以完成对数据验证规则的定义。这样以来,控制层便能判断用户输入、提交数据是否有效,避免了无效数据的录入或者是恶意攻击的可能性。

\noindent
\ttfamily
\hlstd{}\hlkwa{public\ function\ }\hlstd{}\hlkwd{rules}\hlstd{}\hlopt{()}\hspace*{\fill}\\
\hlstd{}\hlopt{\{}\hspace*{\fill}\\
\hlstd{}\hlstd{\ \ \ \ }\hlstd{}\hlkwa{return\ array}\hlstd{}\hlopt{(}\hspace*{\fill}\\
\hlstd{}\hlstd{\ \ \ \ \ \ \ \ }\hlstd{}\hlkwa{array}\hlstd{}\hlopt{(}\hlstd{}\hlstr{'name'}\hlstd{}\hlopt{,}\hlstd{}\hlstr{'required'}\hlstd{}\hlopt{),}\hspace*{\fill}\\
\hlstd{}\hlstd{\ \ \ \ \ \ \ \ }\hlstd{}\hlkwa{array}\hlstd{}\hlopt{(}\hlstd{}\hlstr{'is\textunderscore intl,\ is\textunderscore national,\ is\textunderscore provincial,\ is\textunderscore city,\ }\Righttorque\hspace*{\fill}\\
\hlstr{}\hlstd{\ \ \ \ \ \ \ \ }\hlstr{is\textunderscore school,\ is\textunderscore enterprise,\ is\textunderscore NSF,\ is\textunderscore 973,\ is\textunderscore 863,\ }\Righttorque\hspace*{\fill}\\
\hlstr{}\hlstd{\ \ \ \ \ \ \ \ }\hlstr{is\textunderscore NKTRD,\ is\textunderscore DFME,\ is\textunderscore major'}\hlstd{}\hlopt{,\ }\hlstd{}\hlstr{'numerical'}\hlstd{}\hlopt{,\ }\Righttorque\hspace*{\fill}\\
\hlstd{}\hlstd{\ \ \ \ \ \ \ \ }\hlstd{}\hlstr{'integerOnly'}\hlstd{}\hlopt{=$>$}\hlstd{true}\hlopt{),}\hspace*{\fill}\\
\hlstd{}\hlstd{\ \ \ \ \ \ \ \ }\hlstd{}\hlkwa{array}\hlstd{}\hlopt{(}\hlstd{}\hlstr{'name,\ number,\ fund\textunderscore number'}\hlstd{}\hlopt{,\ }\hlstd{}\hlstr{'length'}\hlstd{}\hlopt{,\ }\hlstd{}\hlstr{'max'}\hlstd{}\hlopt{=$>$}\Righttorque\hspace*{\fill}\\
\hlstd{}\hlstd{\ \ \ \ \ \ \ \ }\hlstd{}\hlnum{255}\hlstd{}\hlopt{),}\hspace*{\fill}\\
\hlstd{}\hlstd{\ \ \ \ \ \ \ \ }\hlstd{}\hlkwa{array}\hlstd{}\hlopt{(}\hlstd{}\hlstr{'app\textunderscore fund,\ pass\textunderscore fund'}\hlstd{}\hlopt{,\ }\hlstd{}\hlstr{'length'}\hlstd{}\hlopt{,\ }\hlstd{}\hlstr{'max'}\hlstd{}\hlopt{=$>$}\hlstd{}\hlnum{15}\hlstd{}\hlopt{),}\hspace*{\fill}\\
\hlstd{}\hlstd{\ \ \ \ \ \ \ \ }\hlstd{}\hlkwa{array}\hlstd{}\hlopt{(}\hlstd{}\hlstr{'start\textunderscore date,\ deadline\textunderscore date,\ conclude\textunderscore date,\ }\Righttorque\hspace*{\fill}\\
\hlstr{}\hlstd{\ \ \ \ \ \ \ \ }\hlstr{app\textunderscore date,\ pass\textunderscore date'}\hlstd{}\hlopt{,\ }\hlstd{}\hlstr{'safe'}\hlstd{}\hlopt{),}\hspace*{\fill}\\
\hlstd{}\hlstd{\ \ \ \ \ \ \ \ }\hlstd{}\hlkwa{array}\hlstd{}\hlopt{(}\hlstd{}\hlstr{'id,\ name,\ number,\ fund\textunderscore number,\ is\textunderscore intl,\ }\Righttorque\hspace*{\fill}\\
\hlstr{}\hlstd{\ \ \ \ \ \ \ \ }\hlstr{is\textunderscore national,\ is\textunderscore provincial,\ is\textunderscore city,\ is\textunderscore school,\ }\Righttorque\hspace*{\fill}\\
\hlstr{}\hlstd{\ \ \ \ \ \ \ \ }\hlstr{is\textunderscore enterprise,\ is\textunderscore NSF,\ is\textunderscore 973,\ is\textunderscore 863,\ is\textunderscore NKTRD,\ }\Righttorque\hspace*{\fill}\\
\hlstr{}\hlstd{\ \ \ \ \ \ \ \ }\hlstr{is\textunderscore DFME,\ is\textunderscore major,\ start\textunderscore date,\ deadline\textunderscore date,\ }\Righttorque\hspace*{\fill}\\
\hlstr{}\hlstd{\ \ \ \ \ \ \ \ }\hlstr{conclude\textunderscore date,\ app\textunderscore date,\ pass\textunderscore date,\ app\textunderscore fund,\ }\Righttorque\hspace*{\fill}\\
\hlstr{}\hlstd{\ \ \ \ \ \ \ \ }\hlstr{pass\textunderscore fund,\ '}\hlstd{}\hlopt{,\ }\hlstd{}\hlstr{'safe'}\hlstd{}\hlopt{,\ }\hlstd{}\hlstr{'on'}\hlstd{}\hlopt{=$>$}\hlstd{}\hlstr{'search'}\hlstd{}\hlopt{),}\hspace*{\fill}\\
\hlstd{}\hlstd{\ \ \ \ }\hlstd{}\hlopt{);}\hspace*{\fill}\\
\hlstd{}\hlopt{\}}\hlstd{}\hspace*{\fill}\\
\mbox{}
\normalfont
\normalsize

\subsection{搜索与筛选}
\label{methodsearch}
Yii框架中的~CDbCriteria~类代表了一条数据库查询的条件,比如说~MySQL~中的~WHERE~字句、~AND~运算符、~OR~运算符、~ORDER BY~子句等。利用它来创建在搜索与查询中需要添加的条件,生成对应的SQL语句。
利用~CDbCriteria~类提供的compare()方法,可以实现搜索与筛选功能。compare()的作用是添加一个比较条件到最终生成的SQL语句的~WHERE~字句中去。还可以通过设置~CDbCriteria~类的~condition~、~group~、~order~等属性,配置SQL中对应的条件。

CActiveDataProvider~类使用AR的CActiveRecord::findAll()方法, 从数据库中检索信息。它的criteria属性是~CDbCriteria~类的一个实例,能够用来查询多种指定条件。

利用~CDbCriteria~类,实现了科研项目模型类~Project~中的搜索方法~search()~,提供给本系统的控制器层调用。search()~方法将~Project~类的各个属性通过一个~CDbCriteria~实例的~compare()~方法添加到了查询条件中,最后利用这些查询条件作为一个~CActiveDataProvider~类的实例构造函数的参数,传递给这个~CActiveDataProvider~类的实例,并返回给供模型层使用。

% public function search()
% {

% 	$criteria=new CDbCriteria;
% 	$criteria->with=array(
% 		'execute_peoples',
% 		'liability_peoples'
% 	);
% 	$criteria->together=true;
% 	$criteria->group = 't.id';
% 	$criteria->compare('execute_peoples.id',$this->searchExecutePeople,true);
% 	$criteria->compare('liability_peoples.id',$this->searchLiabilityPeople,true);
% 	$criteria->compare('name',$this->name,true);
% 	$criteria->compare('number',$this->number,true);
% 	$criteria->compare('fund_number',$this->fund_number,true);
% 	$criteria->compare('is_intl',$this->is_intl);
% 	$criteria->compare('is_national',$this->is_national);
% 	$criteria->compare('is_provincial',$this->is_provincial);
% 	$criteria->compare('is_city',$this->is_city);
% 	$criteria->compare('is_school',$this->is_school);
% 	$criteria->compare('is_enterprise',$this->is_enterprise);
% 	$criteria->compare('is_NSF',$this->is_NSF);
% 	$criteria->compare('is_973',$this->is_973);
% 	$criteria->compare('is_863',$this->is_863);
% 	$criteria->compare('is_NKTRD',$this->is_NKTRD);
% 	$criteria->compare('is_DFME',$this->is_DFME);
% 	$criteria->compare('is_major',$this->is_major);
% 	$criteria->compare('start_date',$this->start_date,true);
% 	$criteria->compare('deadline_date',$this->deadline_date,true);
% 	$criteria->compare('conclude_date',$this->conclude_date,true);
% 	$criteria->compare('app_date',$this->app_date,true);
% 	$criteria->compare('pass_date',$this->pass_date,true);
% 	$criteria->compare('app_fund',$this->app_fund,true);
% 	$criteria->compare('pass_fund',$this->pass_fund,true);

% 	return new CActiveDataProvider($this, array(
% 		'criteria'=>$criteria,
% 	));
% }
% @TODO search 流程图
\section{视图层实现}
视图层的功能主要是定义一个科研项目数据输入、显示的格式与模版,供控制器层调用。
\subsection{整体布局}

本系统视图层的所有页面的主体页面布局形式如图~\ref{index.png}~所示,其中页面顶部为系统的LOGO以及名称;然后下面是导航栏,导航栏左侧部分显示到系统各个模块的链接,右侧部分显示当前模块当前登录用户名称、可供操作的菜单选项以及登录、登出按钮;然后底部的是footer部分,显示关于本系统的总体介绍;中间部分则为各模块各个功能显示的内容。前三部分(顶部LOGO、导航栏、底部footer)在本系统的每个页面都是相同的,提取出来作为本系统的主题模版,而每次渲染的时候只需要渲染中间部分的内容即可,这样可以提高运行效率和系统样式的一致性。

根据在第~\pageref{projectcreate}~页第~\ref{projectcreate}~节所需要实现的科研项目录入功能,在本系统的视图层实现了分别实现用于批量录入和逐个录入的界面。批量录入与逐个录入界面的实现分别放在~WebRoot/protected/views/projects~目录中的~upload.php~和~create.php~中

\pic[H]{系统整体布局}{width=1.0\textwidth}{index.png}

按照第~\ref{yiiprotocoldir}~节中所提高到的文件与目录规范,应该将本小节中所提到的科研项目视图层的实现,放在~WebRoot/protected/views/projects~目录中,每个界面单独放在该目录的一个文件中。

\subsection{录入界面}
\label{projectcreateview}

批量录入科研项目的界面比较简单,只需要提供一个提示用户选择所需要批量录入的Excel文件,以及一个用于确认上传的按钮即可,如图~\ref{pngimport.png}~所示。

\pic[htbp]{批量录入界面}{width=1.0\textwidth}{pngimport.png}

如图~\ref{create2}~所示逐个录入的界面分别按照录入内容的要求实现了文本输入框、下拉菜单等输入表单,鉴于表单较多,响应的式布局技术\footnote{响应式布局是Ethan Marcotte在2010年5月份提出的一个概念,简而言之,就是一个网站能够兼容多个终端——而不是为每个终端做一个特定的版本。这个概念是为解决移动互联网浏览而诞生的。},在较大屏幕上实现多栏显示,同一行可以显示多个输入控件,提高屏幕的利用率;而在较小的屏幕,如手机的屏幕上,自动地调制为单栏显示,提高本系统在较小屏幕下的使用体验。

\begin{pics}[htbp]{逐个录入界面}{create2}
\addsubpic{较大屏幕上多栏显示}{width=0.5\textwidth}{pngcreatedesktop.png}
\addsubpic{较小屏幕上自适应为单栏显示}{width=0.2\textwidth}{pngcreatemobile.png}
\end{pics}

另外指的提到的是,在逐个录入科研项目的过程中,用户需要选择科研项目的参与人员,此时如果仅仅显示一个下拉菜单供用户进行人员的选择,由于人员较多,选择起来不方便且容易出错。因此,如图~\ref{pngselect2.png}~所示,在录入界面人员输入窗口,实现了一个输入与下拉菜单相结合的输入控件,用户可以直接进行选择,也可通过拼音首字母,缩小选择的范围,快速、方便、人性化地进行录入。

\pic[htbp]{人性化的录入界面}{width=0.8\textwidth}{pngselect2.png}


\subsection{修改界面}

修改界面实际上和根据在第~\pageref{projectcreateview}~页第~\ref{projectcreateview}~节所实现的录入界面是一样,只是在相应的输入控件中,显示当前修改科研项目条目已有的内容,如图\ref{pngupdate.png}~
所示。修改界面的实现分别放在~WebRoot/protected/views/projects~目录中的~update.php~中

\pic[htbp]{修改界面}{width=1.0\textwidth}{pngupdate.png}

\subsection{管理界面}
分析第~\pageref{projectupdate}~页第~\ref{projectupdate}~节中提出的修改与删除功能,第~\pageref{projectsearch}~页第~\ref{projectsearch}~节中提出的查询功能可以发现,通常在用户的使用本系统过程中,是按照~\ref{pngadmin.png}~所示的流程来使用上述功能的,也就是说修改、删除、查询三个功能经常需要一起使用,所以在这里将这三个功能集中到一个管理页面上,如图~\ref{pngadmin.png}~所示。用户通过点击“筛选与查找按钮”,可以弹出如图~\ref{pngsearch.png}~所示筛选表单,供用户选择,从而查询,筛选出目标项目;然后点击对应的按钮,查看其详细内容、进行修改、进行删除。管理界面的实现分别放在~WebRoot/protected/views/projects~目录中的~admin.php~中

\pic[htbp]{管理界面}{width=1.0\textwidth}{pngadmin.png}

\pic[htbp]{筛选界面}{width=1.0\textwidth}{pngsearch.png}

\subsection{显示界面}

对外显示:如图~\ref{pngindex.png}~所示,采用表格的形式对角色为游客的用户进行显示。对外显示界面的实现分别放在~WebRoot/protected/views/projects~目录中的~index.php~中

\pic[H]{对外显示界面}{width=1.0\textwidth}{pngindex.png}

对内显示:如图~\ref{pngindex.png}~,即通过上一节中所实现的管理页面对角色为超级管理员、科研项目模块管理员的用户进行显示,在每一条科研项目条目,提供三个按钮,分别对应查看其详细内容、进行修改、进行删除,如图~\ref{pngview.png}~所示。对内显示界面的实现分别放在~WebRoot/protected/views/projects~目录中的~admin.php~中



\pic[H]{对内显示界面}{width=1.0\textwidth}{pngview.png}

\section{控制器层实现}

在在科研项目控制层中,主要完成调用科研项目模型层提供的增加、删除、修改、查询接口,获取对应的数据,将它们按照视图层中
的模版显示出来,以实现在第~\pageref{projectdemand}~页第~\ref{projectdemand}~节需求分析中提出的各项功能。

按照第~\ref{yiiprotocoldir}~节中所提高到的文件与目录规范,应该将本小节中所提到的科研项目控制器类的实现,放在~WebRoot/protected/controllers/~目录中的~ProjectController.php~文件中。

\subsection{录入}
考虑到数据录入过程繁琐性,本系统提供了批量录入功能,用户可以上传一个包含科研项目数据的Excel表格,一次性的录入多个数据,方便了用户的使用。又考虑到后续使用过程中,单个增加项目的需求,本系统同时实现了单个录入功能,能够逐个地向系统添加数据。


\label{controllercreate}

\subsubsection{批量录入}
当用户点击“导入科研项目”时,会最终调用到控制器~ProjectController~的动作~actionImport~,在这个动作中实现了逐个录入功能。程序的流程图如图~\ref{import.pdf}~所示。

在实现批量录入功能时使用到的一个名为~PHPExcel~的第三方库,它的功能是将~Microsoft Excel~文件转换为一个PHP的二维数组。通过使用~PHPExcel~的这个功能,在本系统中,遍历这个数组的每一行,对于每一行数据,新例化一个科研项目模型类,对相应的属性进行赋值,最后调用模型类的Save方法,便可以将Excel中的科研项目的数据逐个存储到数据库中去。另外,同样需要存储科研项目的参与人员,即科研项目与人员之间的关系。此时,在处理那个二维数组每一行的过程中,遇到需要存储的人员关系,首先需要判断那个人员是否已存储在人员数据库中,若没有,则需要新建一个对应的人员。然后再在科研项目与人员关系数据表中,即:tbl\_project\_people\_excecute 和 tbl\_project\_people\_liability ,存储对应的关系。

\pic[H]{批量录入功能 程序流程图}{width=1.0\textwidth}{import.pdf}

另外,导入数据的Excel表格格式如图~\ref{format.png}~所示,由于表项太多,在这里仅给出了部分表项,完整的表项见附录~\ref{projectimportformat}~。

\pic[H]{导入数据Excel表格格式}{width=1.0\textwidth}{format.png}

\subsubsection{单个录入}

当用户点击“增加科研项目”,Yii框架控制器层的路由功能边将用户点击的~URL~解析出来,然后调用相对应的控制器和动作,即控制器~ProjectController~的动作~actionCreate~,在这个动作中实现了逐个录入功能。程序的流程图如图~\ref{create.pdf}~所示。当首先判断HTTP请求中是否设置在POST报文中设置了键为~Project~的查询字符串,若没有设置,则表示用户未提交新的科研项目条目,返回一个录入界面供用户填写;若设置了,则此次请求是用户在录入界面填写了需要增加的科研项目的对应属性后提交的,分别在数据库中新建一个科研项目的条目到数据库中,并设置对应的关系,以保存科研项目中参与人员的信息。

\pic[H]{单个录入功能 程序流程图}{width=1.0\textwidth}{create.pdf}


\subsection{修改}
当用户点击“修改科研项目”时,会最终调用到控制器~ProjectController~的动作~actionUpdate~,在这个动作中实现了逐个录入功能。修改功能的实现与第~\ref{controllercreate}~节中逐个录入的实现非常类似,只需在显示修改界面时,读入需要的修改科研项目原来的数据,显示在录入界面的输入控件中。

\subsection{删除}
当用户点击“删除科研项目”时,会最终调用到控制器~ProjectController~的动作~actionDelete~,在这个动作中实现了删除某个科研项目的功能功能。程序流程图如图~\ref{delete.pdf}~所示。

\pic[H]{删除功能 程序流程图}{width=1.0\textwidth}{delete.pdf}


\subsection{筛选与查询}
当用户点击“管理科研项目”时,会最终调用到控制器~ProjectController~的动作~actionAdmin~,在这个动作中实现了逐个筛选与查询。程序流程如下:
\begin{enumerate}
\item 例化一个科研项目模型类的实例
\item 按照HTTP请求头部中的查询字符串的键,将HTTP请求头部中的查询字符串的值依次赋给这个实例
\item 调用这个实例的search方法,即在第~\pageref{methodsearch}~页第~\ref{methodsearch}~中实现的方法,返回一个CActiveDataProvider类的实例
\item 将这个CActiveDataProvider类的实例传递给视图层,视图层调用这个实例的~getData~方法,返回所有满足步骤2中筛选条件的研项目模型类的实例,并将它们显示出来。
\end{enumerate}

本系统实现了单个条件以及组合条件进行筛选与查询,例如按日期、按维护人员、按参与者、按时间、按级别等等进行搜索与查询,或者按以上几个条件的组合进行查询。下面以按时间以及级别筛选为例,给出筛选功能的实现,其它条件的实现类似,在这里不再赘述。

\begin{enumerate}
\item 如图~\ref{pngprojectsearchimpl.png}~所示,用户希望筛选2010年以后,级别为国家级的项目
\item 通过点击筛选按钮,浏览器发送了如下的查询字符给科研项目控制器(~PaperController~)的筛选与查询动作:\\
%Project[start_date]:>2010-01-01
%Project[is_national]:1
\noindent
\ttfamily
\hlstd{Project{[}start\textunderscore date{]}:}\hlkwa{$>$}\hlstd{}\hlnum{2010}\hlstd{{-}}\hlnum{01}\hlstd{{-}}\hlnum{01}\hspace*{\fill}\\
\hlstd{Project{[}is\textunderscore national{]}:}\hlnum{1}\hlstd{}\hspace*{\fill}
\mbox{}
\normalfont
\normalsize
\item 科研项目控制器的筛选与查询动作(actionSearch)解析这些查询参数,例化一个科研项目模型的实例,将这些条件添加进去
\item 模型类根据这些条件生成SQL语句里WHERE字句的条件:\\
%SELECT * FROM \textasciigrave tbl_project\textasciigrave  \textasciigrave t\textasciigrave  WHERE ((is_national=1) AND (start_date>'2010-01-01'))
\noindent
\ttfamily
\hlstd{SELECT\ {*}\ FROM\ \textasciigrave tbl\textunderscore project\textasciigrave \ \textasciigrave t\textasciigrave \ WHERE\ ((is\textunderscore national=}\hlnum{1}\hlstd{)\ AND\ (\Righttorque\hspace*{\fill}\\
start\textunderscore date}\hlkwa{$>$}\hlstd{'}\hlnum{2010}\hlstd{{-}}\hlnum{01}\hlstd{{-}}\hlnum{01}\hlstd{'))}\hspace*{\fill}
\mbox{}
\normalfont
\normalsize
\item 执行上一步生成的SQL,从数据库中取出数据,返回给科研项目控制器的筛选与查询动作
\item 该动作根据管理视图的模版,将这些数据渲染出来,如图~\ref{pngprojectsearchimplresult.png}~所示
\end{enumerate} 
%pngprojectsearchimpl.png
\pic[H]{多条件筛选界面}{width=1.0\textwidth}{pngprojectsearchimpl.png}
\pic[H]{多条件筛选结果}{width=1.0\textwidth}{pngprojectsearchimplresult.png}

%pngprojectsearchimplresult.png

% \section{本章小结}
% 在本章中,给出了实现本系统详细设计与实现的过程,阐述了开发环境的搭建和配置、数据库的设计与实现、视图层的设计与实现以及模型层的设计与实现。