% !Mode:: "TeX:UTF-8"

\chapter{技术基础}
本章主要介绍在本系统的设计与实现过程用会用到的一些技术、软件工具以及设计模式。
\section{B/S结构}

\gls{B/S}结构\citeup{shuchun2000design}即浏览器和服务器结构。它是随着Internet技术的兴起,对\gls{C/S}结构的一种变化或者改进的结构。在这种结构下,用户工作界面是通过WWW浏览器来实现,极少部分事务逻辑在前端(Browser)实现,但是主要事务逻辑在服务器端(Server)实现,形成所谓三层结构。

\section{AJAX技术}

\gls{AJAX},即“Asynchronous JavaScript and \gls{XML}”(异步的JavaScript与XML技术),指的是一套综合了多项技术的浏览器端网页开发技术。AJAX的概念由Jesse James Garrett所提出\citeup{garrett2005ajax}。

传统的Web应用允许用户端填写表单(form),当提交表单时就向Web服务器发送一个请求。服务器接收并处理传来的表单,然后送回一个新的网页,但这个做法浪费了许多带宽,因为在前后两个页面中的大部分HTML码往往是相同的。由于每次应用的沟通都需要向服务器发送请求,应用的回应时间依赖于服务器的回应时间。这导致了用户界面的回应比本机应用慢得多。

与此不同,AJAX应用可以仅向服务器发送并取回必须的数据,并在客户端采用JavaScript处理来自服务器的回应。因为在服务器和浏览器之间交换的数据大量减少(大约只有原来的5\%),服务器回应更快了。同时,很多的处理工作可以在发出请求的客户端机器上完成,因此Web服务器的负荷也减少了。


\section{面向对象程序设计}

在本系统的设计与实现过程中,用到了许多面向对象程序设计的思想与方法,故在这里对其进行介绍。

面向对象程序设计(Object-oriented programming,缩写:OOP)是一种程序设计范型,同时也是一种程序开发的方法。对象指的是类的实例。它将对象作为程序的基本单元,将程序和数据封装其中,以提高软件的重用性、灵活性和扩展性\cite{rentsch1982object}。下面介绍了面向对象程序设计中的一些概念:
\begin{enumerate}
\item 类:定义了一件事物的抽象特点。通常来说,类定义了事物的属性和它可以做到的(它的行为)。
\item 对象:是类的实例。
\item 继承:是指,在某种情况下,一个类会有“子类”。
\item 多态:是指由继承而产生的相关的不同的类,其对象对同一消息会做出不同的响应。
\item 方法:也称为成员函数,是指对象上的操作,作为类声明的一部分来定义。方法定义了可以对一个对象执行那些操作。
\end{enumerate}




\section{LAMP}

\label{lamp}
\gls{LAMP}是指一组通常一起使用来运行动态网站或者服务器的自由软件名称首字母缩写:

\begin{enumerate}
\item Linux,操作系统
\item Apache,网页服务器
\item MySQL,数据库管理系统
\item PHP,脚本语言
\end{enumerate}

\subsection{服务器操作系统 Linux}

Linux作为本系统服务器的操作系统,主要有开源免费、良好的生态系统、更好的性能几点优势。

Linux 内核源代码可以免费下载。大多数 Linux 发布版本,包括 GNU/Linux 的发行版本和商业的发行版本几乎都提供免费下载服务。免费意味着零试用成本,也不需要为安装在第二台机器上付费。

Linux 作为服务器的优势是,它目前具有最好的生态系统,服务器端的各种软件都为它而设计,默认都认为你是在 Linux上运行,Apache、MySQL等软件随能够在Windows下运行,但是性能却显著地比在Linux下运行时低\citeup{ramana2005some}。

\subsection{网页服务器 Apache}
Apache HTTP Server(简称Apache)是Apache软件基金会的一个开放源代码的网页服务器,可以在大多数计算机操作系统中运行,由于其跨平台和安全性被广泛使用,是最 流行的Web服务器端软件之一\citeup{apache}。它快速、可靠并且可通过简单的API扩充,将Perl、PHP、Python等解释器编译到服务器中。
\subsection{数据库管理系统 MySQL}
MySQL是一个开放源代码的\gls{RDBMS},即关系数据库管理系统,MySQL性能高、成本低、可靠性好,已经成为最流行的开源数据库,因此被广泛地应用在Internet上的中小型网站中。
\subsection{脚本语言 PHP}
\gls{PHP},即“PHP:超文本预处理器”,是一种开源的通用计算机脚本语言,尤其适用于网络开发并可嵌入HTML中使用\citeup{php}。PHP的语法借鉴吸收了C语言、Java和Perl等流行计算机语言的特点,易于一般程序员学习。PHP的主要目标是允许网络开发人员快速编写动态页面,但PHP也被用于其他很多领域。
\subsubsection{PHP框架}

PHP框架提供了一个用以构建web应用的基本框架,从而简化了用PHP编写web应用程序的流程。换言之,PHP框架有助于促进快速应用开发,不但节省开发时间、有助于建立更稳定的应用,而且减少了重复编码的开发。通过确保适当的数据库交换和在表现层编码,框架还可以帮助初学者建立更稳定的应用服务。这可以让你花更多的时间去创建实际的Web应用程序,而不是花时间写重复的代码。

PHP框架的作用相当于\gls{MVC}。MVC是种编程的架构模式,将业务逻辑从UI中分离出来,允许一个一个单独修改(也称为关注点分离)。在MVC中,Model指数据,View指表现层,Controller则指应用程序或业务逻辑。基本上, MVC打破了一个应用的开发进程,这样各组件就可以不受影响地各自工作。从本质上讲,这使得用PHP编码更快更简单。

\subsubsection{Yii框架}

Yii 是一个基于组件、用于开发大型 Web 应用的高性能 PHP 框架。它将 Web 编程中的可重用性发挥到极致,能够显著加速开发进程。Yii代表简单(easy)、高效(efficient)、可扩展(extensible)。Yii框架的的特点:

\begin{enumerate}
\item Yii是一个纯 OOP 框架。对于想使用 Yii 的开发者而言,熟悉面向对象编程(OOP)会使开发更加轻松。
\item Yii 是一个通用 Web 编程框架,能够开发任何类型的 Web 应用。它是轻量级的,又装配了很好很强大的缓存组件,因此尤其适合开发大流量的应用,比如门户、论坛、内容管理系统(CMS)、电子商务系统,等等。
\item Yii 以性能优异、功能丰富、文档清晰而胜出其它框架。它从一开始就为严谨的 Web 应用开发而精心设计,不是某个项目的副产品或第三方代码的组合,而是融合了作者丰富的 Web 应用开发经验和其它热门 Web 编程框架(或应用)优秀思想的结晶。
\end{enumerate}

\section{HTTP协议}
在本系统中,使用了\gls{HTTP}的GET方法和POST方法中的查询字符串(Query String)来完成递客户端与服务端的数据交互,在此对这些概念作简要的介绍。

超文本传输协议(HTTP)的设计目的是保证客户机与服务器之间的通信。HTTP 的工作方式是客户机与服务器之间的请求-应答协议\cite{fielding1999rfc}。

在客户机和服务器之间进行请求-响应时,两种最常被用到的方法是:GET 和 POST。
\begin{itemize}
\item GET~:从指定的资源请求数据。在GET方法中,查询字符串(键/值对)是在 GET 请求的 URL 中发送的,例如:index.php?name1=value1\&name2=value2~
\item POST~:向指定的资源提交要被处理的数据,查询字符串(键/值对)是在 POST 请求的 HTTP 消息主体中发送的,例如:\\
POST index.php HTTP/1.1\\
Host: example.com\\
name1=value1\&name2=value2\\
\end{itemize}

% \section{本章小结}
% 在本章中,主要介绍了在本系统的设计与实现用到的B/S结构、AJAX技术、面向对象程序设计方法、LAMP体系以及HTTP协议,完成了本系统设计过程中的技术选型。