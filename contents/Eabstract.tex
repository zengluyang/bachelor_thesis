% !Mode:: "TeX:UTF-8"
% \begin{Cabstract}{信息管理系统}{B/S架构}{动态网站}{PHP}{MySQL}
% 中文摘要
% \end{Cabstract}

% 随着互联网的飞速发展和社会信息化水平的不断提高,互联网在现代社会中的方方面面起着越来越重要的作用。同时,大学中科研团队在信息管理方面,有着内部信息管理和对外交流展示的两大需求。因此,在本课题中设计并实现了一个团队网络信息管理系统,能提高了科研团队内交流、管理的效率,并能通过对系统中的数据的搜索、筛选、分析,给科研团队的决策提供数据上的支撑;同时利用这些数据,为科研团队的对外交流提供了一个展示的平台。

% 在本文中,首先介绍了在设计与实现本系统过程中使用到的相关技术基础。随后分析了选择这些技术来实现本系统的优点和原因。随后阐述了MVC架构的特点以及如何应用到Web应用中去。随后使用AJAX技术、PHP Yii框架以及MySQL数据库,设计了一个基于B/S结构的团队信息管理系统,并描述了它的整体架构和具体实现。最后进行对本系统进行了功能测试与性能测试,验证了本系统的设计与实现。

\begin{Eabstract}{Informtaion Management System}{B/S Framework}{Dynamic Website}{PHP}{MySQL}

With the rapid development of the Internet and the elevating progress of the informationization in soceity, the Internet has been gaining its influnce in nearly every aspect in modern society. Meanwhile, research groups in universities have two major demands in informtaion management: internal information management and external information display and communication. Therefore, in this project, a Informtaion Management System for research groups is designed and implemented, which can improve the effeciency in intercommunion and management in research groups, and can ultilize these data to be searched, filtered, and analysed, providing statistical support for decision making in research groups; meanwhile, these data are used to provide a platform for external communication and display.

In this thesis, technologies used in designing and implementing this system is firstly discussed. Then details about why and how those technologies are chosen is analysed. Then the features of MVC architecture
and how to use MVC in the web application are discussed. After that, using AJAX, PHP, and the Yii framework, an information management system based on B/S architecture and its overall structure is designed, and its implementation details are provided. Finally,  functional test and performance test are both conducted, verifying the the design and implementation of the system, making sure that the system is designed and implemented properly and correctly.
\end{Eabstract}
